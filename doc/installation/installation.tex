\def\ArtHome[#1]{{\tt <ARTISYNTH\_HOME>#1}}

\title{ArtiSynth Installation Guide for \SYSTEM{}}
\author{John Lloyd, Sebastian Kazenbroot-Guppy, and Antonio S{\'a}nchez}
\setpubdate{Last updated: January, 2020}
\iflatexml
\date{}
\fi

\newif\ifNeedLibraryPath
\NeedLibraryPathfalse

\begin{document}

\maketitle

\iflatexml{\large\pubdate}\fi

\tableofcontents

\section{Introduction}

This document describes how to install and run ArtiSynth
on \FULLSYSTEM{} machines. There are two ways to obtain ArtiSynth:
installing a precompiled release, or installing the latest development
version from Github. Installing a precompiled release is the easiest
way to try out some of the basic demo programs. Installing the
development version is recommended for developers who want to keep
their codebase current and be able to install new features and bug
fixes.

%The typical install sequence looks like this:
%
%\begin{description}
%
%\item[Download]\mbox{}
%
%Download either a precompiled release
%(Section \ref{PrecompiledRelease}), or install the latest development
%version (Section \ref{GitClone}).
%
%\item[Compile]\mbox{}
%
%If you have installed the latest development version, you will also
%have to compile it (Section \ref{Compilation}).  For precompiled
%releases, this step is not needed to run the demonstration models, but
%compilation will be required if you modify any of the models or add
%your own.
%
%\item[Run]\mbox{}
%
%Start ArtiSynth and run the demonstration models (Section \ref{Running}).
%
%\end{description}

Generally, users will also want to install and run external models and
packages that have been created either by others or by themselves.  In
particular, the package collection {\tt artisynth\_models} contains an
open source set of models primarily related to head and neck
anatomy. Installation of this and other packages is discussed in
Section \ref{AdditionalModelsAndPackages}.

\section{Prerequisites}

To install ArtiSynth on \SYSTEM{}, you will need:

\begin{itemize}

\item A 64 bit version of \SYSTEM{}

\item Java 8 (see Section \ref{InstallingJava})

\ifLinux
\item Linux systems require GNU libc version 2.17 or higher
\fi

\item A three-button mouse is recommended for GUI interaction

\item A machine with a good graphics card and a decent
amount of memory (16 Gbytes at least; 32 or more is better).

\end{itemize}

\begin{sideblock}
We have stopped supporting 32 bit systems, for ease of maintenance and
because ArtiSynth applications often require more memory than 32 bit
systems can provide.
\end{sideblock}

\begin{sideblock}
{\bf Important: Eclipse 2020-03 compile bug.} If you are using the Eclipse
IDE (Section \ref{EclipseIDE}), note that Eclipse 2020-03 has a bug
that prevents {\tt artisynth\_core} from compuling properly. Either
install the workaround described in Section \ref{EclipseCompileBug},
or use a different version of Eclipse.
\end{sideblock}

In this document, the location of the ArtiSynth installation \directory{}
will be denoted by \ArtHome[]. For example
if ArtiSynth is installed in
\ifWindows
\begin{verbatim}
  C:\people\roger\artisynth_core
\end{verbatim}
then \ArtHome[] denotes this \directory{}
and \ArtHome[\SEP lib] denotes the sub-\directory{}
\begin{verbatim}
  C:\people\roger\artisynth_core\lib
\end{verbatim}
\else % not Windows
\begin{verbatim}
  /home/roger/artisynth_core
\end{verbatim}
then \ArtHome[] denotes this \directory{}
and \ArtHome[\SEP lib] denotes the sub-\directory{}
\begin{verbatim}
  /home/roger/artisynth_core/lib
\end{verbatim}
\fi % end not Windows

\section{Installing Java}
\label{InstallingJava}

By default, ArtiSynth is compiled to be compliant with Java 8. While
it may be possible to run ArtiSynth under later Java versions, there
have been reports of compatibility problems and warnings involving the
Java OpenGL (JOGL) interface. Therefore we recommend using Java 8
until these issues are resolved.

ArtiSynth requires that you have a full Java development kit (JDK)
installed, which comes with a Java compiler. A simple run time
environment (JRE) will not be sufficient. We recommend Java SE
Development Kit 8u{\it XXX} (where {\it XXX} is the latest revision
number), which can be obtained from Oracle (registration required) at
\begin{quote}
\href{http://www.oracle.com/technetwork/java/javase/downloads/index.html}%
{http://www.oracle.com/technetwork/java/javase/downloads/index.html}.
\end{quote}

\subsection{Making the JDK visible to your system}
\label{MakingJDKVisible}

After the JDK has been installed, it is important to ensure that it is
visible to your system and that it supersedes any other Java
installations. One test for this is to open a
\ifWindows
{\tt CMD} window
\else
terminal window
\fi
and run the command 
\begin{verbatim}
 > javac -version
\end{verbatim}
The output should match the version of the installed JDK. If it does
not, or if the command {\tt javac} is not found,
\ifMacOS
then you can set the ``default'' JDK by setting the {\tt JAVA\_HOME}
environment variable.  This can be done inside the
initialization file for whichever command line shell you are using.

Assume that the desired JDK has version number {\tt 1.8.0\_221} and
that your home \directory{} is {\tt <HOMEDIR>}.  Then for the {\tt
bash} shell, one can use a plain text editor to edit {\tt
<HOMEDIR>/.bashrc} and insert a line of the form
\begin{verbatim}
  export JAVA_HOME=`/usr/libexec/java_home -v 1.8.0_221`
\end{verbatim}
while for the {\tt csh} or {\tt tcsh} shells, one can edit {\tt
<HOMEDIR>/.cshrc} and insert a line of the form
\begin{verbatim}
  setenv JAVA_HOME `/usr/libexec/java_home -v 1.8.0_221`
\end{verbatim}

Setting {\tt JAVA\_HOME} can also be done directly within the shell;
doing it within the initialization file simply avoids the need to do
so each time a new terminal window is opened.
\else % not MacOS
then one fix is to add the \directory{} {\tt <JDK\_DIR>\SEP bin} to
your system Path, as described in Section \ref{SettingPath}, where
{\tt <JDK\_DIR>} is the JDK installation \directory{}. 
\ifWindows
On \SYSTEM{}, {\tt <JDK\_DIR>} is likely to be located under {\tt
Program Files\SEP Java}.
\fi
In particular, {\tt <JDK\_DIR>\SEP bin} should be added {\it ahead} of
any other Java installations that might be specified on the Path. To
see the current contents of the Path, open a
\ifWindows
{\tt CMD} window
\else
terminal window
\fi
and run the command
\ifWindows
\begin{verbatim}
 > echo %PATH%
\end{verbatim}
\else
\begin{verbatim}
 > echo $PATH
\end{verbatim}
\fi
\fi % end not MacOS

\begin{sideblock}
It may not be necessary to 
\ifMacOS
set {\tt JAVA\_HOME}
\else
add {\tt <JDK\_DIR>\SEP bin} to your system Path
\fi
if you are intending to compile and run ArtiSynth exclusively
within an integrated development environment (IDE), such as Eclipse,
since you should be able to specify the JDK directly within the IDE
(as described in Section \ref{ConfiguringEclipse}).
\end{sideblock}

\section{Installing a Precompiled Release}
\label{PrecompiledRelease}

Installing one of the precompiled releases is the easiest way to
obtain ArtiSynth for running demo programs or some existing models.
To do this, go to \href{http://www.artisynth.org/downloads}%
{www.artisynth.org/downloads}, download the distribution you want, and
unzip it in an appropriate location on your computer.

\ifWindows
\begin{sideblock}
On \SYSTEM{}, it is recommended that ArtiSynth be
installed in a location for which none of the \directory{} names
contain spaces (i.e., {\tt Program Files}).  This can be important for
some legacy ArtiSynth programs to run correctly.
\end{sideblock}
\else\fi % end Windows

Once ArtiSynth is downloaded and unzipped, it should be possible to
run it immediately by executing the
\ifWindows
{\tt artisynth.bat} file located in \ArtHome[\SEP bin]
(see Section \ref{artisynthBat}).
\else
{\tt artisynth} command located in \ArtHome[\SEP bin]
(see Section \ref{artisynthCommandLine}).
\fi
More details on running ArtiSynth and its demo models are given in
Section \ref{Running}.

If you modify any of the demonstration models, or add models of your
own, it will be necessary to compile the changes. Compilation is
discussed in Section \ref{Compilation}. 

\section{Installing from Github}
\label{GitClone}

If you wish to obtain updates and bug fixes, we recommend installing
the current development version from Github, which is a web-based
repository service based on the source control management system
Git. A very brief summary of Git is given in Section \ref{GitSummary}.

The latest ArtiSynth development version is available from Github at the URL
\begin{verbatim}
   https://github.com/artisynth/artisynth_core.git
\end{verbatim}
Users who install from Github can continue to update their codebase to
keep it current (Section \ref{UpdatingArtiSynth}).  In some cases,
developers we work with closely can also obtain, by mutual
arrangement, write access to our Github repository, allowing them to
also commit changes.

\begin{sideblock}
Users who have a Github account combined with SSH keys may instead
wish to clone using the SSH URL
\begin{verbatim}
   git@github.com:artisynth/artisynth_core.git
\end{verbatim}
For users with repository write access, this will allow them to
perform subsequent {\tt push} operations without having to
enter a username and password.
\end{sideblock}

Installing from Github entails the following steps:

\begin{itemize}

\item Clone (i.e., checkout) the ArtiSynth repository
(Section \ref{Cloning}).

\item Download the Java and native libraries 
(Section \ref{DownloadingLibraries}).

\item Compile the codebase (Section \ref{Compilation}).

\end{itemize}

It should then be possible to run ArtiSynth and its demo models as
described in Section \ref{Running}.

\subsection{Cloning the repository}
\label{Cloning}

There are several ways to clone ArtiSynth from Github.

\ifWindows

\subsubsection{Clone using Git for Windows}
\label{CloningUsingGitForWindows}

Install Git for Windows (Section \ref{GitForWindows}), and then from
either Git Bash or the {\tt CMD} console, enter the command
%
\begin{lstlisting}[]
  git clone https://github.com/artisynth/artisynth_core.git [<dir>]
\end{lstlisting}
%
The argument {\tt <dir>} is optional and gives the name of the
\directory{} into which the repository and working copy should be
extracted; if omitted, the \directory{} will be named {\tt
artisynth\_core}.

\subsubsection{Cloning using Cygwin}

If you have Cygwin (Section \ref{Cygwin}) installed, along with the
{\tt git} package, then you can check out ArtiSynth within a Cygwin
shell window using the same {\tt git clone} command described
in Section \ref{CloningUsingGitForWindows}.

\else % not Windows
\subsubsection{Cloning using the command line}

Assuming your \SYSTEM{} distribution has Git installed, then you can
clone ArtiSynth from Github using the following command:
%
\begin{lstlisting}[]
 > git clone https://github.com/artisynth/artisynth_core.git [<dir>]
\end{lstlisting}
%
The argument {\tt <dir>} is optional and gives the name of the
\directory{} into which the repository and working copy should be
extracted; if omitted, the \directory{} will be named {\tt
artisynth\_core}.
\fi % end not Windows

\subsubsection{Cloning using Eclipse}
\label{ArtiSynthEclipseCheckout}

If you are planning to develop ArtiSynth models in Java, and if you
are planning to do this with the Eclipse IDE
(Section \ref{EclipseIDE}), then it might be easier to do the Git
clone directly in Eclipse.  Follow the instructions in
Section \ref{importingFromGit}, using the URL {\tt
https://github.com/artisynth/artisynth\_core.git} described above.

\subsection{Downloading the libraries}
\label{DownloadingLibraries}

Because the {\tt jar} files and native libraries used by ArtiSynth
are large, they are not stored in the Github repository.
Instead, they must be downloaded separately. This can be
done using the command \updateArtisynthLibs, located
in \ArtHome[\SEP bin].
\ifWindows
You can execute it by double-clicking on it in a file-browser, or
by opening a command console ({\tt CMD}), navigating
to \ArtHome[], and entering the command
\begin{verbatim}
 % bin\updateArtisynthLibs
\end{verbatim}
\else % not Windows
You can execute it from the command line like this:
\begin{verbatim}
 > cd <ARTISYNTH_HOME>
 > bin/updateArtisynthLibs
\end{verbatim}
\fi % end not Windows

\ifWindows
If you have Cygwin (Section \ref{Cygwin}) installed, 
the same command is available as {\tt
updateArtisynthLibs}, also located in \ArtHome[\SEP bin].
\else\fi % end Windows

\section{Compiling ArtiSynth}
\label{Compilation}

Versions of ArtiSynth obtained from Github need to be compiled before
they can be run. Precompiled releases do not need to be compiled in
order to run the demonstration models, but will need to be compiled if
models are modified or new ones are added.

Java compilation and code development is typically done using an
integrated development environment (IDE), although it is possible
(particularly on Linux and MacOS) to use external text editors and
command line tools. This section describes how to compile
ArtiSynth using either the Eclipse IDE, or
\ifWindows
command line tools available in the Cygwin Unix emulator.
\else
shell-based command line tools.
\fi
For more information on Eclipse and how to obtain it, see
Section \ref{EclipseIDE}.

\subsection{Compiling with Eclipse}
\label{CompilingWithEclipse}

\begin{sideblock}
Note: Eclipse 2020-03 has a bug that prevents {\tt artisynth\_core}
from compuling properly. Either install the workaround described in
Section \ref{EclipseCompileBug}, or use a different version of
Eclipse.
\end{sideblock}

Before using Eclipse to work with ArtiSynth projects, you should
follow some of the basic configuration steps described in
Section \ref{ConfiguringEclipse}.

If your Github clone has been done outside of Eclipse (i.e.,
not according to Section \ref{ArtiSynthEclipseCheckout}), then you
need to first import ArtiSynth into Eclipse. Follow the instructions
in Section \ref{importingExternalProjects}, using 
\ArtHome[] as the top-level project \directory{}
{\tt <PROJECT\_DIR>}.

Once ArtiSynth has been imported, you should be able to compile it.  If
necessary, first open a Java perspective by choosing {\sf Window >
Open Perspective > Java}. The project {\tt artisynth\_core} (or
whatever you might have named it) should appear in the {\sf Package
Explorer} window. To compile the system, select the project in the {\sf
Package Explorer} window, and then choose {\sf Project > Build
Project}. Note that it may be necessary to deselect {\sf Build
Automatically} in order to enable {\sf Build Project}.

\ifWindows
\subsection{Compiling with Cygwin}
\label{CompilingWithCygwin}

If you have Cygwin (Section \ref{Cygwin}) installed, 
ArtiSynth can also be built from a Cygwin terminal by running
a {\tt make} command in the top level \directory{}. Before doing this,
you need to first set the environment variables {\tt ARTISYNTH\_HOME}
and {\tt CLASSPATH} as described in Sections
\ref{EnvironmentVariables} and \ref{CygwinEnvironmentSettings}.
ArtiSynth can then be built by executing

\begin{verbatim}
 > cd <ARTISYNTH_HOME>
 > make
\end{verbatim}
\else % not Windows
\subsection{Compiling from the command line}
\label{CompilingWithCygwin}

ArtiSynth can also be built by running a {\tt make} command in the top
level \directory{}. Before doing this, you need to first set the environment
variables {\tt ARTISYNTH\_HOME} and {\tt CLASSPATH} as described in
Sections \ref{EnvironmentVariables}. ArtiSynth can then be built by
executing

\begin{verbatim}
 > cd <ARTISYNTH_HOME>
 > make
\end{verbatim}
\fi % end not Windows

\section{Running ArtiSynth}
\label{Running}

\ifWindows
\subsection{Running using artisynth.bat}
\label{artisynthBat}

The most direct way to start ArtiSynth is to run the batch file
{\tt artisynth.bat}, located in \ArtHome[\SEP bin].  This can be
done by double-clicking on it in a file browser.

You can also create a shortcut to this batch file (by right clicking
on it and selecting {\sf Create Shortcut}), and then placing this shortcut
in either the {\sf START} menu or on the Desktop. However, the batch file
itself must remain in \ArtHome[\SEP bin].

Alternatively, you can open a command console ({\tt CMD}), navigate
to \ArtHome[], and enter the command
%
\begin{verbatim}
 % bin\artisynth
\end{verbatim}
%
It is recommended to place \ArtHome[\SEP bin] 
in your {\tt PATH} environment variable (Section \ref{EnvironmentVariables}),
so that the console command simplifies to
\begin{verbatim}
 % artisynth
\end{verbatim}
regardless of the current \directory{}.

\begin{sideblock}
{\bf Note:}\\ If the {\tt bin} directory for your {\tt JDK}
installation is not in your systems's Path, the {\tt java} command may
not be visible and this will cause {\tt artisynth.bat} to fail.
See \ref{MakingJDKVisible} for a description of how to fix this.
\end{sideblock}

\subsection{Running using Cygwin}

If you have Cygwin (Section \ref{Cygwin}) installed, 
then you can run ArtiSynth directly with the {\tt artisynth} command
located in \ArtHome[/bin]:
\begin{verbatim}
 > cd <ARTISYNTH_HOME>
 > bin/artisynth
\end{verbatim}
If \ArtHome[/bin] has been placed in your {\tt PATH}
variable (Section \ref{EnvironmentVariables}), then it
is sufficient to do
\begin{verbatim}
 > artisynth
\end{verbatim}
{\tt artisynth} can be called with a number of options; to see these, run
\begin{verbatim}
 > artisynth -help
\end{verbatim}

\else % not Windows
\subsection{Running from the command line}
\label{artisynthCommandLine}

The most direct way to start ArtiSynth is to run the command
\ArtHome[\SEP bin\SEP artisynth]:
%
\begin{verbatim}
 > cd <ARTISYNTH_HOME>
 > bin/artisynth
\end{verbatim}
It is recommended to place \ArtHome[/bin] in your {\tt PATH}
environment variable (Section \ref{EnvironmentVariables}), so
that the command simplifies to
\begin{verbatim}
 > artisynth
\end{verbatim}
regardless of the current \directory{}.

\begin{sideblock}
{\bf Note:}\\ If the {\tt bin} directory for your {\tt JDK}
installation is not in your systems's Path, the {\tt java} command may
not be visible and this will cause the {\tt artisynth} command to fail.
See \ref{MakingJDKVisible} for a description of how to fix this.
\end{sideblock}

\ifMacOS
\subsection{Running from the file browser}

You can also run ArtiSynth from a file browser by double clicking on
\begin{verbatim}
  <ARTISYNTH_HOME>/bin/artisynth.command
\end{verbatim}
Note that {\tt artisynth.command} is just a copy of {\tt artisynth};
the {\tt .command} suffix makes it recognizable to the MacOS GUI as a
command.
\fi % end MacOS
\fi % end not Windows

\subsection{Running using Eclipse}

Once ArtiSynth has been imported into Eclipse (and built if
necessary), it should contain a launch configuration called {\tt
ArtiSynth} that will allow ArtiSynth to be run by choosing {\sf Run >
Run}.

In some cases, one may wish to adjust environment variables, command
line arguments, or Java JVM arguments to affect how ArtiSynth behaves.
Instructions for doing so are contained in Sections
\ref{EclipseEnvironmentVariables} and \ref{EclipseCommandArguments}.

\subsection{Command line arguments}
\label{CommandLineArguments}

The {\tt artisynth} command accepts command line arguments, a full
list of which can be seen by running {\tt artisynth} with the {\tt
-help} option:
\ifWindows
\begin{verbatim}
 % artisynth -help
\end{verbatim}
\else % not Windows
\begin{verbatim}
 > artisynth -help
\end{verbatim}
\fi % end not Windows
Descriptions of these options appear in various places within the
ArtiSynth documentation. For example, one commonly used option is
{\tt -model <modelClassName>}, which instructs ArtiSynth to preload a
model associated with a given class name:
\ifWindows
\begin{verbatim}
 % artisynth -model artisynth.demos.mech.SpringMeshDemo
\end{verbatim}
\else % not Windows
\begin{verbatim}
 > artisynth -model artisynth.demos.mech.SpringMeshDemo
\end{verbatim}
\fi % end not Windows

When running under Eclipse, command line arguments can be set in the
launch configuration, as described in
Section \ref{EclipseCommandArguments}.

\subsection{Loading and Running Models}

Once ArtiSynth starts up, you can use it to load and run
models. General instructions on how to load and run models are given
in the section ``Loading and Simulating Models'' of the
\artisynthManual{uiguide}{ArtiSynth User Interface Guide}. 

By default, ArtiSynth comes with a number of demonstration models,
which can be loaded and run as follows:

From the menu bar, Select {\sf Models > Demos}.  This will display a
submenu of demonstration models. Choosing one will cause that model to
be loaded and displayed in the viewer.  Simulation of the model can
then be started, paused, single-stepped, or reset using the play
controls (Figure \ref{PlayControlsFig}) located at the upper right of
the main ArtiSynth window.

Comprehensive information on exploring and interacting with models is
given in the
\artisynthManual{uiguide}{ArtiSynth User Interface Guide}.

\begin{figure}[h]
\begin{center}
\iflatexml
\includegraphics[]{../uiguide/images/playControls}
\else
\includegraphics[width=2.5in]{../uiguide/images/playControls}
\fi
\end{center}
\caption{The ArtiSynth play controls. From left to right: step size
control, current simulation time, and the reset, skip-back,
play/pause, single-step and skip-forward buttons.}%
\label{PlayControlsFig}
\end{figure}

\section{Installing {\tt artisynth\_models} and Other External Packages}
\label{AdditionalModelsAndPackages}

Typically, an ArtiSynth developer will want to use external models and
packages that exist outside of {\tt artisynth\_core}.  Some of these
may be obtained from external sources.  For example, {\tt
artisynth\_models} is an open source collection of anatomical models,
focused primarily on the head and neck region
(see \href{https://www.artisynth.org/models}{www.artisynth.org/models}).

Installing external models and packages requires a sequence of
operations similar to that for installing ArtiSynth itself:

\begin{enumerate}

\item Downloading (either a precompiled version or repository checkout;
Section \ref{DownloadingCollections}).

\item Compiling (if necessary; Section \ref{CompilingCollections}).

\item Making classes visible to ArtiSynth (Section \ref{MakingVisible}).

\end{enumerate}

\subsection{Downloading}
\label{DownloadingCollections}

Model and package collections are usually available from either Git or
Subversion repositories. 

For {\tt artisynth\_models}:

\begin{itemize}

\item A precompiled version is available
from \href{https://www.artisynth.org/models}{www.artisynth.org/models}.
To ensure compatibility, this should only be used in combination with
the matching precompiled ArtiSynth version. For example, {\tt
artisynth\_models\_3.5.zip} should only be used with {\tt
artisynth\_core\_3.5.zip}.

\item The current development version is available
from Github at
\begin{verbatim}
   https://github.com/artisynth/artisynth_models.git
\end{verbatim}
To ensure compatibility, this should only be used in combination with
the most recent development version of ArtiSynth.

\end{itemize}

Precompiled collections can simply be downloaded and unpacked into a
desired location. Those available from Git or Subversion may be
obtained as described in Sections \ref{GitSummary}
or \ref{SubversionSummary}.  For convenience, we recommend placing
each collection in a \directory{} adjacent to \ArtHome[].

Repository-based collections that contain Eclipse project files (such
as {\tt artisynth\_models}) can be imported directly into Eclipse as
described in Section \ref{importingFromGit} (Git) or
Section \ref{importingFromSubversion} (Subversion).  Other collections
that contain the project files bundled inside an {\tt
eclipseSettings.zip} file may be imported as described in
Section \ref{cloningFromGit} (Git) or
Section \ref{importingFromSubversion} (Subversion).

\subsection{Compiling}
\label{CompilingCollections}

Collections that are obtained from Git or Subversion will need to be
compiled. Compilation will also be necessary if models in the
collection are modified or if new ones are added.

\subsubsection{Compiling with Eclipse}

If compilation is being done with Eclipse, the collection will need to
be imported into Eclipse as a project (if this has not already been
done) by following the instructions in
Section \ref{importingExternalProjects}.  Compilation can then be
performed as described in Section
\ref{CompilingWithEclipse}.

\begin{sideblock}
Important: for collection projects to compile properly in Eclipse, the
{\tt artisynth\_core} project (and any other projects they depend on)
will have to be added to their build path. The default Eclipse
settings supplied with some projects may already contain the required
build path dependencies. For example, the settings for {\tt
artisynth\_models} contain the required reference to {\tt
artisynth\_core}.  In other cases, it may be necessary to add projects
to the build path explicitly, as described
in \ref{AddingProjectsToBuildPath}.
\end{sideblock}

\subsubsection{Compiling from the command line}

If the collection has a {\tt Makefile} in its root
\directory{}, then it can be compiled from
\ifWindows
a Cygwin terminal by running {\tt make} in the root \directory{},
\else
the command line by running {\tt make} in the root \directory{},
\fi
as described in Section \ref{CompilingWithCygwin}.
Before doing this, the top-level \directory{} for the collection's {\tt
class} files must to be added to the {\tt CLASSPATH} environment
variable
\ifWindows
(Sections \ref{EnvironmentVariables} and \ref{CygwinEnvironmentSettings}).
\else
(Section \ref{EnvironmentVariables}).
\fi
In collections maintained by ArtiSynth, this will be the \directory{} {\tt
classes}, located directly under the collection root \directory{} (e.g.,
{\tt artisynth\_models\SEP classes}).

\subsection{Making Classes Visible to ArtiSynth}
\label{MakingVisible}

Models in an external collection are executed by running ArtiSynth
itself (Section
\ref{Running}). However, the classes associated with these
models must be made visible to ArtiSynth. This can be
arranged in several different ways:

\subsubsection{Using the Eclipse Classpath}

If you are running from Eclipse, then you can make the classes of a
collection visible to ArtiSynth by adding its associated Eclipse
project to the {\sf Classpath} of your ArtiSynth launch configuration,
as described in Section \ref{AddingProjectsToLaunch}.

\subsubsection{Using the EXTCLASSPATH file}

Alternatively, you can make the classes of external projects visible
to ArtiSynth by adding the path names of all their top-level class
\directories{} (or {\tt jar} files, if relevant) to the file 
\ArtHome[\SEP EXTCLASSPATH] (described in Section
\ref{EXTCLASSPATHFile}).

For example, suppose the collection {\tt artisynth\_models}
has been placed in {\tt \TOP projects\SEP artisynth\_models}.
The top-level class \directory{} for this collection is located
in {\tt artisynth\_models\SEP classes}, and so the following entry
should be placed in the {\tt EXTCLASSPATH} file:

\ifWindows
\begin{verbatim}
C:\projects\artisynth_models\classes
\end{verbatim}
\else % not Windows
\begin{verbatim}
/projects/artisynth_models/classes
\end{verbatim}
\fi % end not Windows

\subsubsection{Using CLASSPATH environment variable}

\ifWindows
Finally, if you are running using {\tt artisynth.bat} (or {\tt
artisynth} in Cygwin), then you can make external classes visible by
adding them to your {\tt CLASSPATH} environment variable (see Sections
\ref{EnvironmentVariables} and \ref{CygwinEnvironmentSettings}).
\else % not Windows
Finally, if you are running from the command line using the {\tt
artisynth} command, then you can make external classes visible by adding
them to your {\tt CLASSPATH} environment variable (see Section
\ref{EnvironmentVariables}).
\fi % end not Windows

\section{Updating ArtiSynth and Other Packages}
\label{UpdatingArtiSynth}

One reason to use a clone of the latest ArtiSynth
development version is to be able to migrate recent changes into your
code base. When a significant update occurs, a posting is made to the
ArtiSynth update log, currently located at
\artisynthManual{updates}{www.artisynth.org/doc/html/updates/updates.html}.
Users may also be notified via the {\tt artisynth-updates} email list.

Users working from Eclipse may update simply by selecting
the project in the {\sf Package Explorer} and selecting {\sf Team >
Pull} from the context menu.

\ifWindows
Updating may also be done by issuing the 
\begin{verbatim}
   > git pull
\end{verbatim}
command from within \ArtHome[], 
using either Git for Windows (Section \ref{GitForWindows}),
the Cygwin shell (Section \ref{Cygwin}), or another Git application.
\else % not Windows 
Updating may also be done from the command line by issuing the 
\begin{verbatim}
   > git pull
\end{verbatim}
command from within the ArtiSynth installation \directory{}.
\fi % end not Windows

Other Git-based packages may be updated similarly. 

For Subversion-based packages, updating is done by issuing an {\tt
update} request (Section \ref{SubversionSummary}). This can be done
from within Eclipse (if a Subversion plugin has been installed;
Section \ref{SubversionPlugIn}) by selecting the project in the {\sf
Package Explorer} and selecting {\sf Team > Update} (or {\sf Team >
Update to Head}) from the context menu. Subversion updates can also be
done
\ifWindows
using either TortoiseSVN (Section \ref{TortoiseSVN}), or from the
Cygwin command line by calling
\else % not Windows 
from the command line by calling
\fi % end not Windows
\begin{verbatim}
   > svn update
\end{verbatim}
from inside the top-level project \directory{}.

\subsection{Library updates}

Occasionally, a software update will be accompanied by a change in the
libraries located in \ArtHome[\SEP libs].  When this happens, it will
be indicated on the ArtiSynth update log and appropriate instructions
will be given. Sometimes, it will be necessary to explicitly update
the libraries after doing the main update. This can be done by
executing {\tt updateArtisynthLibs} as described in
Section \ref{DownloadingLibraries}.

\section{The Eclipse IDE}
\label{EclipseIDE}

Eclipse is an integrated development environment (IDE) commonly used
for Java code development, and many ArtiSynth developers use it for
both developing models in Java and for running the system. This section
describes how to load ArtiSynth projects into Eclipse, and how to
configure it for running ArtiSynth. A general introduction to Eclipse
is beyond the scope of this document, but there are many Eclipse
resources available online.

\subsection{Eclipse 2020-03 compilation bug}
\label{EclipseCompileBug}

\begin{sideblock}
Eclipse 2020-03 has a bug that prevents {\tt artisynth\_core} from
compiling properly:
%
\begin{quote}
\href{https://bugs.eclipse.org/bugs/show_bug.cgi?id=561287}%
{bugs.eclipse.org/bugs/show\_bug.cgi?id=561287}
\end{quote}
%
\end{sideblock}

This should be fixed in subsequent versions of Eclipse. Meanwhile, you
can either install a different version of Eclipse, or use the
following workaround: Add the line
%
\begin{verbatim}
-DmaxCompiledUnitsAtOnce=0
\end{verbatim}
%
to the file {\tt eclipse.ini} in the Eclipse installation directory,
and restart Eclipse.

\subsection{Installing Eclipse}

Eclipse can be obtained from
\href{http://www.eclipse.org/downloads/packages}%
{www.eclipse.org/downloads/packages}.  A
good version to obtain (at the time of this writing) is {\sf Eclipse
IDE for Java Developers}.

\begin{sideblock}
The Eclipse instructions described below are based on the ``Neon''
distribution, but should be largely similar for later versions.
\end{sideblock}

\subsection{Configuring Eclipse for ArtiSynth}
\label{ConfiguringEclipse}

There are a few things one should do to configure Eclipse for
ArtiSynth related projects:

\begin{itemize}

\item {\it Open a Java perspective}, by choosing {\sf Window > Open Perspective >
Java}. This will place Eclipse in a state suitable for editing Java
projects. Individual projects can be viewed and navigated through the
{\sf Package Explorer} window.

\item {\it Check that Java is ArtiSynth compatible:}
The Java version used by Eclipse needs to be set to one that is
ArtiSynth compatible (which is at present Java 8).  To verify that
this is the case, choose {\sf Window > Preferences > Java > Installed
JREs} and verify that a compatible Java is installed and selected. If
an appropriate JRE is not installed, it can be added by clicking the
{\sf Add...} button on the right of the panel, and then finding the
Java 8 distribution on your system.

\item {\it Configure building to prevent excess resource copying:}
The steps required to do this are described in Section
\ref{PreventingResourceCopying}, and will speed up project
compilation by preventing unneeded files from being copied into the
{\tt classes} \directory{}.

\end{itemize}

\subsection{Importing ArtiSynth projects into Eclipse}

ArtiSynth projects include the core distribution ({\tt
artisynth\_core}), the open source models collection {\tt
artisynth\_models} (which contains human anatomy models), as well as
other model and code collections maintained by the ArtiSynth team and
other users.

There are several ways to import ArtiSynth projects into Eclipse.  If
the project has already been downloaded or checked out from a
repository, then it can be imported as an external project
(Section \ref{importingExternalProjects}). Otherwise, Eclipse itself
may sometimes be used to checkout the project directly from either Git
or Subversion (Sections \ref{importingFromGit},
\ref{cloningFromGit}, and \ref{importingFromSubversion}).

\subsubsection{Importing external projects}
\label{importingExternalProjects}

Let {\tt <PROJECT\_DIR>} denote the top-level project \directory{}.
For the core distribution {\tt artisynth\_core}, this will also be
\ArtHome[].

First check the following:

\begin{itemize}

\item {\tt <PROJECT\_DIR>} should contain the
Eclipse project files (including {\tt .project}). If it does not, it
should contain a file {\tt eclipseSettings.zip}, which should be
unzipped directly into {\tt <PROJECT\_DIR>} (not into a
sub-\directory{}), as described in
Section \ref{installingProjectFiles}, so that {\tt .project} and {\tt
.classpath} then appear there.

\item If the project is the ArtiSynth core distribution (i.e., 
{\tt artisynth\_core}), and was obtained from Github, then make sure
you have downloaded the required {\tt jar} files and native libraries
as described in Section \ref{DownloadingLibraries}.

\end{itemize}

Then import the project into Eclipse as follows:

\begin{enumerate}

\item From within Eclipse, choose {\sf File > Import ...}.

\item An {\sf Import} dialog will appear. 
Select {\sf General > Existing Projects into Workspace} and click {\sf Next}.

\item An {\sf Import Projects} dialog will appear. 
In the field {\sf Select root directory}, enter (or browse to) the
{\it parent} \directory{} of {\tt <PROJECT\_DIR>}. The project
itself should now appear in the {\sf Projects} box (Figure
\ref{EclipseImportProjects:fig}). (If other projects are contained in the 
parent \directory{}, these will appear as well.)
Make sure that the
desired project is selected and then click {\sf Finish}.

\end{enumerate}

\begin{sideblock}
If Eclipse complains that {\sf "No projects are found to import"}, or
does not otherwise show the project as available for import,
then most likely the {\tt <PROJECT\_DIR>}
\directory{} does not contain a {\tt .project} file.
In the case of repositories that keep Eclipse project
files bundled in am {\tt eclipseSettings.zip} file, this usually
means that the contents of that file were not properly unzipped into
{\tt <PROJECT\_DIR>} (Section \ref{installingProjectFiles}).
\end{sideblock}

\begin{figure}
\begin{center}
\iflatexml
   \includegraphics{images/EclipseImportProjectsLinuxSmall}
\else
   \includegraphics[width=0.4\textwidth]{images/EclipseImportProjectsLinux}
\fi   
\end{center}
\caption{Eclipse Import Projects dialog.}%
\label{EclipseImportProjects:fig}
\end{figure}

\subsubsection{Importing projects from a remote Git repository}
\label{importingFromGit}

Some source repositories contain Eclipse project files in their
repositories, and so can be imported directly into Eclipse using the
repository's URL. The Eclipse project associated with ArtiSynth is
called {\tt artisynth\_core}, while the models project is called {\tt
artisynth\_models}.

To import these directly:

\begin{enumerate}

\item If necessary, open a Java perspective by choosing {\sf Window >
Open Perspective > Java}.

\item Choose {\sf File > Import... > Git > Projects from Git}
from the main menu.

\item A {\sf Select Repository Source} dialog will appear. Choose
{\sf Clone URI} and click {\sf Next}.

\item A {\sf Source Git Repository} dialog will appear
(Figure \ref{EclipseGitImport:fig}, left).  Under {\sf Location},
enter the project URI in the {\sf URL} field.  The default URL for
ArtiSynth is {\tt https://github.com/artisynth/artisynth\_core.git}. In
some cases, such as when using an SSH URL, or accessing a project with
restricted access, it may also be necessary to provide a user name and
password in the {\sf Authentication} fields. (This will be your Gitlab
or Github account information for repositories stored on those sites.)
After entering the required information, click {\sf Next}.

\item A {\sf Branch Selection} dialog may appear. If it does,
make sure only the {\sf master} branch is selected, and then click {\sf Next}.

\item A {\sf Local Destination} dialog will appear (Figure
\ref{EclipseGitImport:fig}, right). In the {\sf Directory} field, enter
the path of the local \directory{}, which will contain both the cloned
repository and the working copy.  For ArtiSynth itself, this will also
be the ArtiSynth home \directory{} (\ArtHome[]).  After
entering the \directory{} information, click {\sf Next}.

\item A {\sf Select a wizard ...} dialog will appear.
Select {\sf Import existing Eclipse projects} and click {\sf Next}.

\item An {\sf Import Projects} dialog will appear.
Make sure project you wish to import is selected and click {\sf
Finish}.

\item In the case of ArtiSynth (i.e., {\tt artisynth\_core}), from {\it outside}
Eclipse, download the Java and native libraries, as described in
Section \ref{DownloadingLibraries}. Then refresh the project from
within Eclipse (by selecting it in the {\sf Package Explorer} window
and choosing {\sf Refresh} from the context menu. The project should
now be able to compile.

\end{enumerate}

\begin{figure}[h]
\begin{center}
\begin{tabular}{cc}
\iflatexml
   \includegraphics{images/EclipseSourceGitRepositorySmall} &
   \includegraphics{images/EclipseLocalDestinationSmall}
\else
   \includegraphics[width=0.45\textwidth]{images/EclipseSourceGitRepository} &
   \includegraphics[width=0.45\textwidth]{images/EclipseLocalDestination}
\fi   
\end{tabular}
\end{center}
\caption{Eclipse dialogs for importing a Git repository.}%
\label{EclipseGitImport:fig}
\end{figure}

\subsubsection{Cloning a project from a remote Git repository}
\label{cloningFromGit}

Some source repositories (such as {\tt artisynth\_research}) do not
directly contain Eclipse project files in their repositories.
Instead, the project files is contained inside an {\tt
eclipseSettings.zip} file that must be extracted into the project root
\directory{}. This is to prevent undesired local changes to the project
settings from being propagated to all users.

In this case, we proceed as follows:

\begin{enumerate}

\item Choose {\sf Window > Show View > Other ... > Git > Git Repositories} 
from the main menu to open a {\sf Git Repositories} view window.

\item Within the {\sf Git Repositories} window, choose the
button (or pull down menu item) that says {\sf Clone a repository}.

\item 
A {\sf Source Git Repository} dialog will appear (Figure
\ref{EclipseGitImport:fig}, left). Enter the URL for the
repository. Also, if the repository has read access restrictions, it
will generally be necessary to specify a user name and password in the
{\sf Authentication} fields.  (This will be your Gitlab or Github
account information for repositories stored on those sites.)  After
entering the required information, click {\sf Next}.

\item A {\sf Branch Selection} dialog may appear. Usually
you want to select only the {\sf master} branch, and then click {\sf
Next}.

\item A {\sf Local Destination} dialog will appear (Figure
\ref{EclipseGitImport:fig}, right). In the {\sf Directory} field, enter
the path of the local \directory{}, which will contain both the cloned
repository and the working copy. After entering the \directory{}
information, click {\sf Finish}.

\item Finally, from {\it outside} Eclipse, locate the file
{\tt eclipseSettings.zip} in the project's top \directory{}, and then
unzip this file directly into that \directory{} (not into a
sub-\directory{}), so that {\tt .project} and {\tt .classpath} appear
in the top \directory{}.
\ifMacOS
MacOS users are strongly encouraged to read  
Section \ref{installingProjectFiles} for details on how to do this.
\else
For details, see Section \ref{installingProjectFiles}.
\fi % end MacOS

\end{enumerate}

The project can now be imported into Eclipse by following the steps in
Section \ref{importingExternalProjects}, using the project's
parent \directory{} as the ``root directory''.

\subsubsection{Importing from a Subversion repository}
\label{importingFromSubversion}

If Eclipse has a Subversion plug-in installed (Section
\ref{SubversionPlugIn}), you may import an ArtiSynth project by
checking it out directly from the repository located by the project's
{\it Subversion\_URL}. For the project {\tt artisynth\_projects}, this
is
\begin{verbatim}
  https://svn.artisynth.org/svn/artisynth_models/trunk
\end{verbatim}
Other projects will have different URLs.

The following instructions assume the Subversive plug-in.

\begin{enumerate}

\item Choose {\sf File > Import} from the main menu, select {\sf SVN >
Project from SVN} and click {\sf Next}.

\item You now need to specify a repository location, as specified by a
{\it Subversion\_URL}.  If you've previously done an SVN checkout, a
menu will appear allowing you to select a previously used URL. If one
of these is sufficient, select it and click {\sf Next} to go to Step
4. Otherwise, select {\sf Create a new repository location} and click
{\sf Next} to enter a repository dialog. If no previous locations are
known this dialog will appear automatically.

\item If you are specifying a new location in the repository dialog:

\begin{itemize}

\item Under the {\sf General} tab, enter the {\it Subversion\_URL} in the
{\sf URL} box. If you are just checking out the trunk of the
repository (i.e., if your Subversion URL ends in {\tt /trunk}), then
you should omit the final {\tt /trunk} since this is selectable in Step 4.

\item If you are checking out a repository that is not available for
anonymous access, or if you need write access to the repository, enter
the appropriate user name and password in the {\sf Authentication}
section of the dialog. (If the SVN repository is hosted by us, we will
have given you this name and password.) You will probably want to
check {\sf Save authentication} as well.

%\item If you are just checking out the trunk of the repository (i.e.,
%if your Subversion URL ends in {\tt /trunk}, as most examples in this
%guide do), then go to the {\sf Advanced} tab and uncheck {\sf Enable
%Structure Detection}.

\item Click {\sf Next}.

\end{itemize}

\item In the {\sf Select Resource} dialog, use the {\sf URL} selector
box to select the full URL to be used for the checkout. If you are
just checking out the trunk of the repository, then choose {\tt
Subversion\_URL/trunk} which should be available as a selection.

\item Click {\sf Finish}

\item In the {\sf Check Out As} dialog, select {\sf Check out as a
project with name specified}, adjust the project name if desired,
and click {\sf Next}.

\item Specify the location for the check out. If you leave {\sf Use
default workspace location} selected, this will be {\tt
workspace/project\_name}, where {\tt workspace} is the Eclipse
workspace \directory{} and {\tt project\_name} is the project name
selected in the previous step. Otherwise, you can specify an explicit
checkout location (which does not have to be located in the Eclipse
workspace). For ArtiSynth core checkouts, the project name is
typically {\tt artisynth\_core} and the the checkout location will
become the ArtiSynth install \directory{} \ArtHome[].

\item Click {\sf Finish}.

\item If necessary, open a Java perspective by choosing {\sf Window >
Open Perspective > Java}. The project should appear in the {\sf
Package Explorer} window.

\item From {\it outside} Eclipse, check to see if the file
{\tt eclipseSettings.zip} exists in the project's top \directory{}.
If it does, unzip this file directly into the top \directory{} (not
into a sub-\directory{}), so that {\tt .project} and {\tt .classpath}
appear there.
\ifMacOS
MacOS users are strongly encouraged to read  
Section \ref{installingProjectFiles} for details on how to do this.
\else
For details, see Section \ref{installingProjectFiles}.
\fi % end MacOS

\item Finally, load the new settings into the project by selecting the
project in the {\sf Package Explorer} window and selecting {\sf
Refresh} from the context menu.

\end{enumerate}

\subsubsection{Installing project files}
\label{installingProjectFiles}

Some project repositories contain their eclipse project files bundled
in the zip file {\tt eclipseSettings.zip}, instead of keeping them
under direct repository control. This is to prevent unwanted local
configuration changes from being propagated back into the repository.
The project files need to be unzipped directly into the project
\directory{} (not into a sub-\directory{}) to enable the project to be
loaded into Eclipse.

Let {\tt <PROJECT\_DIR>} denote the top-level project \directory{}.
%
\ifWindows
On Windows, project files can be extracted from within the 
file browser. Double click on {\tt eclipseSettings.zip}
and extract the files into {\tt <PROJECT\_DIR>}.
\else % not Windows 
Project files can be extracted using the command line.
Open a command shell, 
switch to the {\tt <PROJECT\_DIR>} \directory{}, and run {\tt unzip}:
\begin{verbatim}
  > cd <PROJECT_ROOT>
  > unzip eclipseSettings.zip
\end{verbatim}
\fi % end not Windows
This will create the files {\tt .project} and {\tt .classpath}, along
with the \directory{} {\tt .settings}, in {\tt <PROJECT\_DIR>}.  In
the case of {\tt artisynth\_core}, it will also create the file {\tt
ArtiSynth.launch} containing the default launch configuration.

\begin{sideblock}
Note: if unzip queries about overwriting {\tt .project}, answer [y]es.
\end{sideblock}

\ifMacOS
\begin{sideblock}
{\bf Attention MacOS users:}\\[0.5em]
While it is possible to unzip files from the file browser by clicking
on {\tt eclipseSettings.zip} and then extracting directly, the default
zip utility on MacOS will create a new sub-folder called {\tt
eclipseSettings} and will extract the files there.
\emph{You do not want this!!}
Some of the files are then labeled as ``hidden'' by MacOS, which will
prevent you from moving them to the correct place manually. 
Either extract the files using the command line as described
above, or use a more standard application like {\sf 7-Zip} ({\sf 7zX} for OSX).
\end{sideblock}
\fi % end MacOS

\subsection{Configuring environment variables}
\label{EclipseEnvironmentVariables}

While it is generally {\it not} necessary to set environment variables
in Eclipse, it may be useful to do this on occasion to control certain
aspects of ArtiSynth's operation.  Directions on setting the
environment variables are given in
Section \ref{SettingEnvironmentVariables}, and descriptions of the
variables themselves may be found in
Section \ref{EnvironmentVariables}.

Some variables that are commonly set within Eclipse include:

\begin{itemize}

\item {\tt ARTISYNTH\_HOME}: If set, this should be set to
\ArtHome[]. Normally ArtiSynth is able to infer its own location
internally, so it is generally unnecessary to set this variable
explicitly.

\item {\tt OMP\_NUM\_THREADS}: Specifies the maximum number of processor cores
available for multicore execution.

\item {\tt ARTISYNTH\_PATH}: A list of folders, separated by semi-colons ";", 
which ArtiSynth uses to search for configuration files. 
See Section \ref{EnvironmentVariables}.

\end{itemize}

If any of the above variables have already been set externally in
\SYSTEM{} (\environmentSectionRef), such that they are visible
to Eclipse at start-up, then they do not need to be set in the launch
configuration.

%In addition to setting environment variables, if you are running Java
%1.7 (but not Java 1.8), then you should increase the memory space
%allocated for classes. Do this by adding the following JVM argument to
%your launch configuration:
%%
%\begin{lstlisting}
% -XX:MaxPermSize=100M
%\end{lstlisting}
%%
%Instructions for doing this are given in Section
%\ref{SettingCommandLineArguments}.

\ifNeedLibraryPath

\begin{sideblock}
{\bf Important:} At present, eclipse does not expand environment variables.
In all the variable settings described below, references to 
\ArtHome[]should be expanded (manually) to the path of the
ArtiSynth install \directory{}.
\end{sideblock}

\ifWindows
\begin{sideblock}
{\bf Note:} There is a bug in Eclipse 4.0+ on Windows where it will replace 
the system's native {\tt \%PATH\%} variable rather than appending to it.  
To correct this behavior, define {\tt PATH} as 
{\tt \$\{env\_var:path\};\$ARTISYNTH\_HOME\SEP lib\SEP \ARCH{}} 
\end{sideblock}
\fi % end Windows
\fi

\subsubsection {Setting environment variables}
\label{SettingEnvironmentVariables}

To set environment variables within Eclipse:

\begin{enumerate}

\item Open a java perspective if necessary by choosing
  {\sf Window > Open Perspective > Java}.

\item Select the ArtiSynth project in the {\sf Package Explorer} form.

\item Choose {\sf Run > Run Configurations...} to open the {\sf Run
  Configurations} window.

\item In the left panel, under {\sf Java Application}, select 
the launch configuration (the default is named {\sf ArtiSynth}).

\item In the right panel, select the {\sf Environment} tab.

\item To create a new environment variable, click the {\sf New} button and
  enter the name and value in the dialog box. 
%See Figure \ref{EclipseEnvVariables:fig}.

\item When finished, make sure that {\sf Append environment to native
  environment} is selected, and click {\sf Apply}.

\end{enumerate}

%\begin{figure}
%\begin{center}
%\iflatexml
%\includegraphics[]{images/EclipseEnvVariablesSmall}
%\else
%\includegraphics[width=5.0in]{images/EclipseEnvVariables}
%\fi
%\end{center}
%\caption{Setting environment variables within Eclipse.}%
%\label{EclipseEnvVariables:fig}
%\end{figure}

\subsection{Command line and JVM arguments}
\label{EclipseCommandArguments}

As described in Section \ref{CommandLineArguments}, the {\tt artisynth}
command accepts command line arguments. To invoke these when
running from Eclipse, it is necessary to set the desired arguments in
the launch configuration, as described below. 

Sometimes it may also be necessary to set JVM arguments, which control
the Java virtual machine running ArtiSynth.  An example of such an
argument is {\tt -Xmx}, which can be used to increase the maximum
amount of memory available to the application.  For example, {\tt
-Xmx6g} sets the maximum amount of memory to 6 gigabytes.

\subsubsection {Setting command line and JVM arguments}
\label{SettingCommandLineArguments}

To set command line arguments for your Eclipse application:

\begin{enumerate}

\item Open a java perspective if necessary by choosing
  {\sf Window > Open Perspective > Java}.

\item Select the ArtiSynth project in the {\sf Package Explorer} form.

\item Choose {\sf Run > Run Configurations...} to open the {\sf Run
  Configurations} window.

\item In the left panel, under {\sf Java Application}, select
the launch configuration (the default is named {\sf ArtiSynth}).

\item In the right panel, select the {\sf Arguments} tab.

\item Program arguments (which are passed directly to ArtiSynth)
should be specified in the {\sf Program arguments} box.  JVM arguments
should be specified in the {\sf VM arguments} box. See Figure
\ref{EclipseRunArguments:fig}.

\item When finished, click {\sf Close}.

\end{enumerate}

\begin{figure}
\begin{center}
\iflatexml
\includegraphics[]{images/EclipseRunArgumentsSmall}
\else
\includegraphics[width=5.0in]{images/EclipseRunArguments}
\fi
\end{center}
\caption{Setting command line and JVM arguments for a run configuration.}%
\label{EclipseRunArguments:fig}
\end{figure}


\subsection{Adding projects to the build path}
\label{AddingProjectsToBuildPath}

A project imported into Eclipse may depend on the packages and
libraries found in other projects to compile properly.  For example,
ArtiSynth applications which are external to {\tt artisynth\_core}
will nonetheless depend on {\tt artisynth\_core}. To ensure proper
compilation, project dependencies should be added to each dependent
project's build path.

\begin{enumerate}

\item Select the dependent project in the {\sf Package Explorer} form.

\item Right click and choose {\sf Build Path > Configure Build Path...} 

\item In the right panel, select the {\sf Projects} tab.

\item Click the {\sf Add} button, select the project dependencies,
      and click {\sf OK}

\item Click {\sf OK} in the Java Build Path panel

\end{enumerate}

\subsection{Adding projects to the ArtiSynth launch configuration}
\label{AddingProjectsToLaunch}

The classes of external projects can be made visible to ArtiSynth by
adding the projects themselves to the Classpath of the ArtiSynth launch
configuration.

\begin{enumerate}

\item From the main menu, choose {\sf Run > Run Configurations...}
to open a {\sf Run Configurations} dialog.

\item In the left panel, under {\sf Java Application}, select your
ArtiSynth launch configuration (the default one is called {\sf
ArtiSynth}). This may already be selected when you open the panel.

\item In the right panel, select the {\sf Classpath} tab.

\item In the {\sf Classpath} window, select {\sf User Entries},
and then click the {\sf Add Projects} button.

\item In the {\sf Project Selection} dialog, select the external
projects that you wish to add. Generally, the boxes
{\sf Add exported entries ...} and {\sf Add required projects ...}
can be unchecked. Click {\sf OK}.

\item Close the {\sf Run Configurations} dialog.

\end{enumerate}

\subsection{Installing a Subversion plug-in}
\label{SubversionPlugIn}

In order to work with Subversion from within Eclipse, either to check
out ArtiSynth from the repository, or to update or commit changes, it
is necessary to use a Subversion plug-in. First, check to see if your
version of Eclipse contains an Subversion plug-in:

Open an import panel using {\sf File > Import...}, and then look for
{\sf SVN} in the set of available import sources. If you don't see SVN
listed, it will be necessary to install a plug-in.

We recommend the Eclipse-supported Subversive plug-in, but if this
proves difficult for any reason, there are other options, such as
Subclipse, currently obtainable from
\href{http://subclipse.tigris.org/servlets/ProjectProcess?pageID=p4wYuA}%
{subclipse.tigris.org}.

Instructions for installing Subversive can be obtained at
\href{http://www.eclipse.org/subversive/installation-instructions.php}%
{www.eclipse.org/subversive/installation-instructions.php}.

One way to install Subversive is through the Eclipse Marketplace.  If
you have an older version of Eclipse that doesn't have Marketplace,
you may be able to obtain it from
\href{http://www.eclipse.org/mpc/}{www.eclipse.org/mpc}.  To access
the Marketplace, click {\sf Help > Eclipse Marketplace}. Once the
available applications have been displayed, type {\tt Subversive} into
the {\sf Find} box in the top-left corner of the Marketplace
window. Navigate to the package labeled {\sf Subversive - SVN Team
Provider} and click {\sf Install}. On the {\sf Confirm Selected
Features} screen, ensure all boxes are checked and click the button
labeled {\sf Confirm >}. Restart Eclipse when prompted.

One more step is now necessary. Re-open Eclipse, and you should be
prompted to choose an SVN connector in the start menu.  SVN connectors
interface Subversive to the SVN server, and are OS and
server-specific. A recommended SVN Connector will be pre-selected for
downloading; this is most likely the one you need.

If Eclipse did not prompt you to choose a connector when it restarted,
you can install SVN connectors separately (thanks to bmaupin at
Stackoverflow for this information):

\begin{enumerate}

\item  Go to 
\href{http://www.polarion.com/products/svn/subversive/download.php}%
{www.polarion.com/products/svn/subversive/download.php}

\item Under the latest {\sf Release}, copy the Subversive SVN
Connectors URL. The current URL for Eclipse 4.3 Kepler
is \href{http://community.polarion.com/projects/subversive/download/eclipse/3.0/kepler-site/}%
{http://community.polarion.com/projects/subversive/download/eclipse/3.0/kepler-site}.

\item In Eclipse, go to {\sf Help > Install New Software...} and 
click {\sf Add...}  

\item Copy the URL for the Subversive SVN Connectors into the {\sf
Location} box and click {\sf OK}

\item Check {\sf Subversive SVN Connectors}, click {\sf Next}, and
then follow the instructions to complete installation.

\end{enumerate}

If in doubt about the connector you need, you can install multiple
ones, and then adjust the one Subversive actually uses by going to
{\sf Windows > Preferences}, opening {\sf Team > SVN}, and then
opening the {\sf SVN Connector} tab.

\subsection{Preventing excessive resource copying}
\label{PreventingResourceCopying}

By default, ArtiSynth classes are built in a directory tree ({\tt
<PROJECT\_DIR>\SEP classes}) that is separate from the source tree ({\tt
<PROJECT\_DIR>\SEP src}), where {\tt <PROJECT\_DIR>} denotes the project
root \directory{} and is \ArtHome[] for ArtiSynth itself.
That means that Eclipse will try to copy all non-Java files and
\directories{} from the source tree into the build tree. For ArtiSynth,
this is excessive, and results in many files being copied that don't
need to be, since ArtiSynth looks for resources in the source tree
anyway.

It is possible to inhibit most of this copying:

\begin{enumerate}

\item Choose {\sf Window > Preferences} (or {\sf Eclipse > Preferences}).

\item Select {\sf Java > Compiler > Building}.

\item Open {\sf Output folder}, and in the box entitled {\sf Filter resources},
  enter the single character `{\tt *}'.

\end{enumerate}

\section{Additional Information}

\subsection{Adding Directories to the System Path}
\label{SettingPath}

The system ``Path'' is a list of directories which the system searches
in order to find executables. Adding a directory to the path allows
executables contained in that directory to be called directly from a
\ifWindows
command window such as {\tt CMD}.
\else
terminal window.
\fi

\ifWindows

\subsubsection{Windows 10}

\begin{enumerate}

\item Open the {\sf Start} search, enter ``{\tt env}'', and choose
{\sf ``Edit the system environment variables''}.

\item Click on {\sf Environment Variables}.

\item Under {\sf User variables} (the top window), click on {\sf Path}
and click {\sf Edit}. If {\sf Path} does not exist, click {\sf New}.

\item In the {\sf Edit environment variable} dialog, click {\sf New}
and enter the full path name for each directory you wish to add.

\item Close each dialog by clicking {\sf OK}.

\end{enumerate}

\subsubsection{Windows 8 and earlier}

\begin{enumerate}

\item Right-click {\sf My Computer}, and then click {\sf Properties}.

\item Click the {\sf Advanced} tab.

\item Click {\sf Environment variables}.

\item In the top {\sf User variables} window, click on {\sf Path} and 
then {\sf Edit}. If {\sf Path} does not exist, click {\sf New}.

\item In the edit window, add the full path name for each new directory,
separated by semi-colons '{\tt ;}'.

\item Close each dialog by clicking {\sf OK}.

\end{enumerate}

For example, if ArtiSynth is installed at {\tt C:\BKS artisynth\BKS
artisynth\_core} and the desired {\tt JDK} is at {\tt C:\BKS Program
Files\BKS Java\BKS jdk1.8.0\_221}, then we can add the {\tt bin}
directories for both by setting the User path to
\begin{verbatim}
  C:\artisynth\artisynth_core\bin;C:\Program Files\Java\jdk1.8.0_221\bin
\end{verbatim}

\fi % end Windows

\ifWindows\else % not Windows

\ifMacOS
Since \SYSTEM{} is a Unix-based system, 
\else\ifLinux
On Linux,
\fi % end Linux
\fi % end not Windows
directories can be added to the path by appending them to the {\tt
PATH} environment variable, which is a list of directories separated
by colons `{\tt :}'. The most direct way to do this is to redefine
{\tt PATH} inside one of the initialization files for whichever
command line shell you are using.

Assume that your home folder is {\tt <HOMEDIR>}. Then for the {\tt
bash} shell, one can edit {\tt <HOMEDIR>/.bashrc} and insert a line of
the form
\begin{verbatim}
   export PATH=<DIR>:$PATH
\end{verbatim}
while for the {\tt csh} or {\tt tcsh} shells, one can edit {\tt
<HOMEDIR>/.cshrc} and insert a line of the form
\begin{verbatim}
   setenv PATH <DIR>":"$PATH
\end{verbatim}

\ifMacOS
On Mac OS X 10.8 and greater, directories can also be added to the
path by adding a text file containing the directories to {\tt
/etc/paths.d}.  In particular, we can create a file called {\tt
ArtiSynth} in {\tt /etc/paths.d} that contains the full path names of
the desired directories.

\begin{enumerate}

\item Open a terminal window

\item Use {\tt sudo} to create {\tt /etc/paths.d/ArtiSynth} with a plain
text editor. For example:
\begin{verbatim}
  sudo nano /etc/paths.d/ArtiSynth
\end{verbatim}

\item Add the full path name of each desired directory, one per line,
and save the file.

\item To test the revised {\tt PATH}, open a new terminal
window and enter the command: {\tt echo \$PATH}.

\end{enumerate}

\fi % end MacOS
\fi % end not Windows

\begin{sideblock}
Most most command windows and applications need to be restarted in
order to get them to notice changes to the {\tt PATH}.
\end{sideblock}

\ifWindows

\subsection{Git for Windows}
\label{GitForWindows}

A version of Git for Windows can be installed from
\href{https://git-scm.com/downloads}{git-scm.com/downloads}.

It installs a version of Git Bash for that can be used for entering
Git commands.  While going through the install steps, there is an
option to select between:

\begin{enumerate}

\item Use Git from Git Bash only
\item Use Git from the Windows Command Prompt
\item Use Git and optional Unix tools from the Windows Command Prompt

\end{enumerate}

Option 2 (Git from CMD) is often fine for most users, and allows them
to run all git commands directly from a command console {\tt CMD}.  There are
other options for setting the default editor, the default
console to open when starting Git Bash, etc., but the defaults should
work well in most cases.

\subsection{The TortoiseSVN Client}
\label{TortoiseSVN}

As mentioned above, some ArtiSynth models may be distributed via
Subversion (SVN), and access to these will require an SVN client
program. A popular SVN client for Windows is TortoiseSVN, which can be
acquired from
\href{http://tortoisesvn.net}{tortoisesvn.net}. On that website,
navigate to {\sf Downloads} and then select the package appropriate
for your operating system (which presumably will be 64-bits).

Once downloaded, run the MSI installer and follow the instructions in
the installer. When the download completes, right-clicking in any file
browser in Windows should now present you with SVN version control
options, analogous to the command line instructions discussed in other
parts of the documentation. We will briefly discuss the three
most common operations:

\begin{description}

\item[Checking out a repository]

To check out an SVN repository with TortoiseSVN, right-click in any
file exploration window and select {\sf SVN Checkout...} from the
context menu. The current \directory{} will be the default \directory{} for
the {\sf Checkout directory} field, and the {\it Subversion\_URL} can
be selected in the {\sf URL of repository} field. It is recommended
that other fields remain as their default values.

\item[Updating a working copy]

While inside a working copy of a repository in a file browser window,
updating that repository is as simple as right-clicking and selecting
{\sf SVN Update}.

\item[Committing changes]

Users which have write-access to a Subversion repository may commit
changes that they have made.  Right-click in a file browser window
inside a working copy of a repository and select {\sf
Commit...}. Write a commit message and the select {\sf OK}.

\end{description}

\subsection{Cygwin}
\label{Cygwin}

Cygwin provides Windows users with a Linux-like, shell-based command
environment.  Its provides useful tools and enables ArtiSynth users to
employ all the script-based commands in \ArtHome[\SEP bin], as well
as various {\tt Makefile} commands.

Cygwin can be downloaded from 
\href{http://www.cygwin.com}{www.cygwin.com}. Run the
executable that was just downloaded (setup.exe) to begin
installation. After selecting a download source, an install directory,
a package directory, an internet connection, and a download site, the
installer will display a list of packages which the user can select
for download. Normally, a default set of packages is already selected
for installation. It is advisable to install these packages to ensure
that Cygwin retains its basic capabilities. It is also recommended
that you select the following additional packages:

\begin{itemize}

\item {\tt archive}
\item {\tt git}, {\tt svn}, {\tt make} (located under {\tt devel})
\item {\tt python} (located under {\tt interpreters}).
\item {\tt openssh} (located under {\tt net})

\end{itemize}

If you are planning to compile C/C++ code, you may also want to
install the various {\tt gcc} and {\tt gdb} packages (located under {\tt devel}).
\fi % end Windows

\subsection{Environment variables}
\label{EnvironmentVariables}

This is a glossary of all the environment variables that are
associated with building or running ArtiSynth. Often, the system can
detect and set appropriate values for these automatically. In other
cases, as noted in the above documentation, it may be necessary or
desirable for the user to set them explicitly.

\begin{description}

\item[ARTISYNTH\_HOME]\mbox{}
 
The path name of the ArtiSynth installation \directory{}.

\item[ARTISYNTH\_PATH]\mbox{}

A list of \directories{}, separated by \separatorDesc, which ArtiSynth
uses to search for configuration files such as {\tt .artisyntInit} or
{\tt .demoModels}.  A typical setting for {\tt ARTISYNTH\_PATH}
consists of the current \directory{} (indicated by "{\tt .}"), the user's
home \directory{}, and the ArtiSynth installation \directory{}. If {\tt
ARTISYNTH\_PATH} is not defined explicitly in the user's environment,
ArtiSynth assumes an implicit path consisting of the \directory{}
sequence just described.

\item[CLASSPATH]\mbox{}

A list of \directories{} and/or {\tt jar} files, separated by
\separatorDesc, which Java uses to locate its class files. This
variable should be set to include \ArtHome[\SEP classes]
and \ArtHome[\SEP lib\SEP *] (the latter uses the
wildcard {\tt *} to specify all the {\tt jar} files in 
\ArtHome[\SEP lib]).

\item[PATH]\mbox{}
 
A list of \directories{}, separated by \separatorDesc, which the
operating system uses to locate executable programs and
applications. Placing \ArtHome[\SEP bin] in your {\tt PATH} (as
described in Section \ref{SettingPath}) will allow you to run {\tt
artisynth} and related commands directly from a command window.

\ifNeedLibraryPath
\ifWindows\else % not Windows 
\item[\LIBRARYPATH{}]\mbox{}

A list of \directories{}, separated by colons
":", which the operating system searches in order to find shared libraries.
Should be set to include \ArtHome[\SEP lib\SEP \ARCH{}].
\fi % end not Windows
\fi % end NeedLibraryPath

\item[OMP\_NUM\_THREADS]\mbox{}
 
Specifies the maximum number of processor cores that are available for
multicore execution. Setting this variable to the maximum number of
cores on your machine can significantly increase performance.

\end{description}

%JYTHON\_HOME::
%If Jython is installed on the system, should be set to the name of the
%Jython installation \directory{}.

Note that settings for most of the above can be derived from the value
of {\tt ARTISYNTH\_HOME}.

\ifWindows
\subsubsection{Setting environment variables}
\label{settingWindowsVariables}

On Windows, a user can view, set, or change environment variables via
the following steps:

{\bf Windows 10:}

\begin{enumerate}

\item Open the {\sf Start} search, enter ``{\tt env}'', and choose
{\sf ``Edit the system environment variables''}.
\item Click on {\sf Environment Variables}.
\item Choose one of the following options:

\begin{itemize}
\item Click {\sf New} to add a new variable name and value.
\item Click an existing variable, and then {\sf Edit} to change its name or value.
\item Click an existing variable, and then {\sf Delete} to remove it.
\end{itemize}

\item Close each dialog by clicking {\sf OK}.

\end{enumerate}

{\bf Windows 8 and earlier:}

\begin{enumerate}

\item Right-click {\sf My Computer}, and then click {\sf Properties}.
\item Click the {\sf Advanced} tab.
\item Click {\sf Environment variables}.
\item Choose one of the following options:

\begin{itemize}
\item Click {\sf New} to add a new variable name and value.
\item Click an existing variable, and then {\sf Edit} to change its name or value.
\item Click an existing variable, and then {\sf Delete} to remove it.
\end{itemize}

\item Close each dialog by clicking {\sf OK}.

\end{enumerate}

Variable settings can reference other environment variables, by
surrounding them with percent signs, as in {\tt \%VARIABLE\_NAME\%}.  For
example, suppose you already have an environment variable {\tt HOME} that
gives the location of your home \directory{}, and your ArtiSynth
distribution is located in {\tt packages\SEP artisynth\_core} relative to your
home \directory{}. Then the environment variable {\tt ARTISYNTH\_HOME} can be
specified as

\begin{verbatim}
  %HOME%\packages\artisynth_core
\end{verbatim}

\subsubsection{Typical environment settings}
\label{TypicalEnvironment}

Typical settings for the environment variables described above might
look like this:

\begin{lstlisting}[]
ARTISYNTH_HOME c:\users\joe\artisynth_core
ARTISYNTH_PATH .;c:\users\joe;%ARTISYNTH_HOME%
CLASSPATH %ARTISYNTH_HOME%\classes;%ARTISYNTH_HOME%\lib\*
PATH %ARTISYNTH_HOME%\bin;%PATH%
OMP_NUM_THREADS 2
\end{lstlisting}

\subsubsection{Cygwin environment settings}
\label{CygwinEnvironmentSettings}

When running a Cygwin terminal, it is possible to set environment
variables in the startup script for the terminal's shell. Assuming
that this shell is {\tt bash}, then the environment settings described
in \ref{TypicalEnvironment} can be set by inserting the following
into one of {\tt bash}'s initialization files (typically {\tt
\textasciitilde/.bashrc}):

\begin{lstlisting}[]
# set AH to the location of the ArtiSynth install folder
AH=$HOME/artisynth_core
export PATH=$AH/bin:$PATH

# Use Windows path style for ARTISYNTH_HOME, ARTISYNTH_PATH, and CLASSPATH:
export ARTISYNTH_HOME=`cygpath -w $AH`
export ARTISYNTH_PATH=".;`cygpath -w $HOME`;$ARTISYNTH_HOME"
export CLASSPATH="$ARTISYNTH_HOME\classes;$ARTISYNTH_HOME\lib"'\*;'"$CLASSPATH"
export OMP_NUM_THREADS=2
\end{lstlisting}

Note that even on Cygwin, the environment variables {\tt
ARTISYNTH\_HOME}, {\tt ARTISYNTH\_PATH}, and {\tt CLASSPATH} should be
set using Windows path conventions. That is because they may not
necessarily be invoked in a Unix-like context.
\fi % end Windows

\ifWindows\else % not Windows 
\subsubsection{Example environment set up for {\tt bash}}
\label{BashEnvironmentSetup}

If you are using {\tt bash} as your shell, then the environment can be
configured by placing a block of commands similar to the following in
one of your {\tt bash} initialization files (typically {\tt
\textasciitilde/.bashrc}), located in your home \directory{}:

\begin{lstlisting}[]
# set ARTISYNTH_HOME to the appropriate location ...
export ARTISYNTH_HOME=$HOME/artisynth_2_X
export ARTISYNTH_PATH=.:$HOME:$ARTISYNTH_HOME
export CLASSPATH=$ARTISYNTH_HOME/classes:$ARTISYNTH_HOME/lib/'*':$CLASSPATH
export PATH=$ARTISYNTH_HOME/bin:$PATH
# Set to the number of cores on your machine:
export OMP_NUM_THREADS=2 
\end{lstlisting}

Be sure to set {\tt ARTISYNTH\_HOME} to the proper location of your
ArtiSynth installation \directory{}.

These environment variables will be passed on to any program which you
run from the shell (such as {\tt artisynth} or {\tt eclipse}).
\ifMacOS
However, they will {\bf not} be passed on to programs (such as eclipse)
which you launch from the dock.
\fi

Alternatively, you can source the script {\tt setup.bash}, located in
the installation \directory{}:

\begin{verbatim}
 > source setup.bash
\end{verbatim}

This will determine the system type automatically and set the
environment variables accordingly, with {\tt ARTISYNTH\_HOME} set to the
current \directory{} from which the script is called (however,
it {\it won't} set {\tt OMP\_NUM\_THREADS}).

\subsubsection{Example environment setup for {\tt csh} or {\tt tcsh}}
\label{CshEnvironmentSetup}

If you are using {\tt csh} or {\tt tcsh} as your shell, then the
environment can be configured by placing a block of commands similar
to the following in your {\tt .cshrc} file, located in your home
\directory{}:

\ifLinux
\begin{lstlisting}[]
# set ARTISYNTH_HOME to the appropriate location ...
setenv ARTISYNTH_HOME $HOME/artisynth_2_X
setenv ARTISYNTH_PATH .":"$HOME":"$ARTISYNTH_HOME
setenv CLASSPATH "$ARTISYNTH_HOME/classes:$ARTISYNTH_HOME/lib/*:$CLASSPATH"
setenv PATH $ARTISYNTH_HOME/bin":"$PATH
# Set to the number of cores on your machine:
setenv OMP_NUM_THREADS 2 
\end{lstlisting}
\else\ifMacOS
\begin{lstlisting}[]
# set ARTISYNTH_HOME to the appropriate location ...
setenv ARTISYNTH_HOME $HOME/artisynth_2_X
setenv ARTISYNTH_PATH .":"$HOME":"$ARTISYNTH_HOME
setenv CLASSPATH "$ARTISYNTH_HOME/classes:$ARTISYNTH_HOME/lib/*:$CLASSPATH"
setenv PATH $ARTISYNTH_HOME/bin":"$PATH
# Set to the number of cores on your machine:
setenv OMP_NUM_THREADS 2 
\end{lstlisting}
\fi % end MacOS
\fi % end not Linux

These environment variables will be passed on to any program which you
run from the shell (such as {\tt artisynth} or {\tt eclipse}).
\ifMacOS
However, they will {\bf not} be passed on to programs (such as eclipse)
which you launch from the dock.
\fi

Alternatively, you can source the script {\tt setup.csh}, located in
the installation \directory{}:

\begin{verbatim}
 > source setup.csh
\end{verbatim}

This will determine the system type automatically and set the
environment variables accordingly, with {\tt ARTISYNTH\_HOME} set to the
current \directory{} from which the script is called (however,
it {\it won't} set {\tt OMP\_NUM\_THREADS}).
\fi % end not Windows

\subsection{ArtiSynth Libraries}

ArtiSynth uses a set of libraries located under
\ArtHome[\SEP lib].  These include a number of {\tt jar}
files, plus native libraries located in architecture-specific
sub-\directories{} ({\tt \ARCH{}} for \FULLSYSTEM{} systems).

As described in Section \ref{DownloadingLibraries}, these libraries
need to be downloaded automatically if the system is obtained from the
Github repository. The required libraries are listed in the file
\ArtHome[\SEP lib\SEP LIBRARIES]. This file is checked
into the repository, so that the system can always determine what
libraries are needed for a particular checkout version.

Occasionally the libraries are changed or upgraded.  If you run
ArtiSynth with the {\tt -updateLibs} command line option, the program
will ensure that not only are all the required libraries present, but
that they also match the latest versions on the ArtiSynth server.

\subsection{The EXTCLASSPATH File}
\label{EXTCLASSPATHFile}

In order to run an external model or package in ArtiSynth, all class
paths (i.e., class \directories{} or {\tt jar} files) associated with
those external classes must be made visible to ArtiSynth. One way to
do this is to list these class paths as entries in the text file {\tt
EXTCLASSPATH}, located in \ArtHome[].

To add class paths to {\tt EXTCLASSPATH}, open it using a
plain text editor
\ifWindows
(such as {\tt Notepad})
\else
(such as {\tt vim}, {\tt gedit}, or {\tt emacs}),
\fi
and add each required path. For clarity, each path is typically
added on a separate line. However, multiple paths can be
added on the same line if they are separated by the
path separator character used for that OS.

The syntax rules for {\tt EXTCLASSPATH} are:

\begin{enumerate}

\item Class path entries on the same line should be separated by a
path separator character (a semi-colon '{\tt ;}' for Windows
and a colon '{\tt :}' for MacOS and Linux).

\item The {\tt \#} character comments out all remaining characters
to the end of line.

\item The {\tt \$} character can be used to expand environment variables.

\item Any spaces present {\it will} be included in the path name.

\end{enumerate}

An example {\tt EXTCLASSPATH} might look like this:

\ifWindows
\begin{verbatim}
C:\research\artisynth_models\classes
C:\research\models\special.jar
$HOME\projects\crazy\classes
\end{verbatim}
\else % not Windows 
\begin{verbatim}
/research/artisynth_models/classes
/research/models/special.jar
$HOME/projects/crazy/classes
\end{verbatim}
\fi % end not Windows

\subsection{Quick Git Summary}
\label{GitSummary}

Git is a distributed source control management (SCM) system that is
widely used in the software industry.  A full discussion of Git is
beyond the scope of this document, but a large literature is available
online. Generally, when you {\it clone} a Git repository, you create a
local copy of that repository on your machine, along with a checked
out working \directory{} containing the most recent version of the code
(which is referred to as the HEAD).

Unlike client/server SCMs, Git is distributed, with users maintaining
their own private copies of a repository. This allows a great deal of
flexibility in usage, but also adds an extra ``layer'' to the
workflow: when you ``checkout'' from a repository or ``commit'' to it,
you do so with respect to your own {\it local} copy of that
repository, {\it not} the original ({\it origin}) repository from
which you performed the original clone. The process of merging in
changes from the origin to the local repository is known as
``pulling'', while committing changes from the local repository back
to the origin is known as ``pushing''.

There is also another layer of interaction when you commit changes to
the local repository: you first {\it add} them to a staging area
(also known as the ``index''), and then commit them using the {\tt commit}
command.

A very simple workflow for a typical ArtiSynth user is summarized
below. The actions are described in command-line form, but the same
commands can generally be issued through Eclipse or other
interfaces. First, clone the most recent version of the ArtiSynth
repository on Github:

\begin{verbatim}
  git clone https://github.com/artisynth/artisynth_core.git [<dir>]
\end{verbatim}

This will create a local copy of the Github repository, along with a
checked out ``working copy'', in the \directory{} specified by {\tt
<dir>}, or in {\tt artisynth\_core} if {\tt <dir>} is
omitted.  The repository itself will be located in a sub-\directory{}
called {\tt .git}.

Other Git repositories can be cloned in a similar manner.  If the
repository has read access restrictions, then when performing a checkout it
may also be necessary to specify a user name for which the repository
has granted read access. This is typically done by embedding the user
name in the URL, as in (for example)
{\tt https://user@host.xz/path/to/repo.git}.

Later, to fetch the latest updates from the Github repository and
merge them into your working copy, then from within the working copy
\directory{} you can do
\begin{verbatim}
  git pull
\end{verbatim}

If you make changes to some files in your working copy and wish to
commit these to your local repository, you first {\it add} (or remove
them) from the staging area using commands such as:
\begin{verbatim}
  git add <fileName>    # add a new (or modified) file
  git add *             # add all files
  git rm <fileName>     # remove a file
\end{verbatim}
and then commit them to your local repository using
\begin{verbatim}
  git commit -m "commit message"
\end{verbatim}
Note that you can also add modified files and commit them using the single
command
\begin{verbatim}
  git commit -m -a "commit message"
\end{verbatim}

To see the current status of the files in your working copy
and the staging area, use the command
\begin{verbatim}
  git status
\end{verbatim}
and to see the commit history for particular files or \directories{},
use 
\begin{verbatim}
  git log [ <filename> ... ]
\end{verbatim}

Finally, to push your changes back to the Github repository (assuming
you have permission do so), you would do so using the command
\begin{verbatim}
  git push origin master
\end{verbatim}

Note that the above commands all have various options not mentioned.
There are also numerous topics that haven't been discussed, including
the creation and merging of branches, but there are many useful online
resources that describe these in detail. Some current references
include
\begin{verbatim}
   https://git-scm.com/docs
   http://rogerdudler.github.io/git-guide
\end{verbatim}

\subsection{Quick Subversion Summary}
\label{SubversionSummary}

Subversion is a client/server source control management (SCM) system
that is widely used in the software industry.  A full discussion of
Subversion is beyond the scope of this document, but a large
literature is available online. 

Subversion allows you to {\it check out} a codebase from a (often
remote) repository into a local {\it working copy}, {\it update}
recent changes from the repository into the working copy, and (if one
has the appropriate permissions) {\it commit} local changes back into
to repository.

A Subversion {\it client} application is used to access both
Subversion repositories and local working copies. The remainder of
this discussion will assume use of the command-line client {\tt svn},
although other clients are available, including TortoiseSVN for
Windows 
\ifWindows
(Section \ref{TortoiseSVN}) 
\fi
and the Subversion plug-ins for
Eclipse (Section \ref{SubversionPlugIn}).

Some ArtiSynth models collections and code extensions are distributed
through Subversion, including the {\tt artisynth\_projects} package
used by some collaborators. A very simple workflow involving one of
these is summarized below.

First, check out the most recent version from the repository, using
the repository's URL. For example, the URL for {\tt artisynth\_projects}
is {\tt https://svn.artisynth.org/svn/artisynth\_projects},
and the associated checkout command is
\begin{verbatim}
  svn checkout https://svn.artisynth.org/svn/artisynth_projects/trunk [<dir>]
\end{verbatim}
This will create a local working copy of the ``trunk'' branch of {\tt
artisynth\_projects} in the \directory{} specified by {\tt <dir>},
or in {\tt artisynth\_projects} if {\tt <dir>} is omitted. Local
repository information is stored in a sub-\directory{} called {\tt
.svn}.

If the SVN repository has read access restrictions (which 
{\tt artisynth\_projects} actually does), then when performing a
checkout it may also be necessary to specify a user name or email
address for which the repository has granted read access. This may be
done with the {\tt --username} option. The user will also
typically be prompted for an access password.

\begin{sideblock}
{\bf Note:}\\ 
If you omit the trailing {\tt /trunk} from the
Subversion URL, then the checkout will contain the entire Subversion
\directory{} structure, including the subdirectories {\tt trunk}, {\tt
branches}, and {\tt tags}, which is generally not needed by most
users.
\end{sideblock}

Later, to fetch the latest updates from the repository and
merge them into your working copy, from within the local
\directory{}, would you simply do
\begin{verbatim}
  svn update
\end{verbatim}

If you make changes to some files in your working copy and wish to
commit these back to the repository (assuming you have the necessary
permissions), then you can issue the command
\begin{verbatim}
  svn commit -m "commit message"
\end{verbatim}
To add or remove files from the repository, one may use
the commands
\begin{verbatim}
  svn add <fileName> ...     # add files
  svn delete <fileName> ...  # delete files
\end{verbatim}
prior to performing the commit.

To see the current status of the files in your working copy,
use the command
\begin{verbatim}
  svn status
\end{verbatim}
and to see the commit history for particular files or \directories{},
use 
\begin{verbatim}
  svn log [ <filename> ... ]
\end{verbatim}

Note that the above commands all have various options not mentioned.
There are also numerous topics that haven't been discussed, including
the creation and merging of branches, but there are many useful online
resources that describe these in detail. The most comprehensive
is probably the \href{http://svnbook.red-bean.com}{Subversion ``Redbook''}.

\end{document}

%To obtain one of the packaged distributions:
%
%\begin{enumerate}
%
%\item Go to \href{http://www.artisynth.org}{http://www.artisynth.org}
%
%\item Select {\sf Software/Downloads}. This will bring up a page directing you 
%to the {\sf ArtiSynth Download Form}.
%
%\item Select the link to the {\sf ArtiSynth Download Form}.
%
%\item Fill in and submit this form. An email containing a link and a password 
%will automatically be sent to the e-mail address you provide in this form.
%
%\item Open the e-mail mentioned above and click on the given link to find a 
%password protected webpage.
%
%\item Enter the password provided by the email. Take care that no spaces are 
%included, as the system is sensitive to them.
%
%\item This will lead to a different webpage, with links to various ArtiSynth 
%distributions. Select any one to download, and click it.
%
%\item Download the distribution, and unzip it in an appropriate
%location on your computer.
%
%\end{enumerate}
