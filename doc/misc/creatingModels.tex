% Aug 27, 2014: has 118 sections. artisynth.tex and maspack.tex have
%   about 28 and 23 sections, with about 375 words/sec, so we can
%   expect this to be around 45,000 words, or about 120 pages.

\documentclass{article}

% Add search paths for input files
\makeatletter
\def\input@path{{../}{../../}{../texinputs/}}
\makeatother

\usepackage{amsmath}
\usepackage{framed}
%\usepackage{bibtopic}

%\usepackage{layout}
%\usepackage{showframe}
%%
%% Default settings for artisynth
%%
\NeedsTeXFormat{LaTeX2e}
%%\ProvidesPackage{artisynthDoc}[2012/04/05]

\usepackage[T1]{fontenc}
\usepackage[latin1]{inputenc}
\usepackage{listings}
\usepackage{makeidx}
\usepackage{latexml}
\usepackage{graphicx}
\usepackage{framed}
\usepackage{booktabs}
\usepackage{color}

\newcommand{\pubdate}{\today}
\newcommand{\setpubdate}[1]{\renewcommand{\pubdate}{#1}}
\newcommand{\code}[1]{{\tt #1}}

\iflatexml
\usepackage{hyperref}
\setlength\parindent{0pt} 
\else
%% then we are making a PDF, so include things that LaTeXML can't handle: 
%% docbook style, \RaggedRight
\usepackage{ifxetex}
\usepackage{xstring}
\usepackage{pslatex} % fixes fonts; in particular sets a better-fitting \tt font

\usepackage[most]{tcolorbox}
\definecolor{shadecolor}{rgb}{0.95,0.95,0.95}
\tcbset{
    frame code={}
    center title,
    left=0pt,
    right=0pt,
    top=0pt,
    bottom=0pt,
    colback=shadecolor,
    colframe=white,
    width=\dimexpr\textwidth\relax,
    enlarge left by=0mm,
    boxsep=0pt,
    arc=0pt,outer arc=0pt,
}%

\usepackage[A4]{artisynth_papersize}
%\usepackage[letter]{artisynth_papersize}
\usepackage[hyperlink]{asciidoc-dblatex} 

%\usepackage{verbatim}
\usepackage{ragged2e}
\setlength{\RaggedRightRightskip}{0pt plus 4em}
\RaggedRight
\renewcommand{\DBKpubdate}{\pubdate}
\renewcommand{\DBKreleaseinfo}{}
\fi

% set hypertext links to be dark blue:
\definecolor{darkblue}{rgb}{0,0,0.8}
\definecolor{sidebar}{rgb}{0.5,0.5,0.7}
\hypersetup{colorlinks=true,urlcolor=darkblue,linkcolor=darkblue,breaklinks=true}

%%%%%%%%%%%%%%%%%%%%%%%%%%%%%%%%%%%%%%%%%%%%%%%%%%%%%%%%%%%%%%%%%%%%%%%%%%%%%
%
% Define macros for handling javadoc class and method references
%
%%%%%%%%%%%%%%%%%%%%%%%%%%%%%%%%%%%%%%%%%%%%%%%%%%%%%%%%%%%%%%%%%%%%%%%%%%%%%
\makeatletter

% macro to enable line break if inside a PDF file
\def\pdfbreak{\iflatexml\else\\\fi}

% code inspired by http://stackoverflow.com/questions/2457780/latex-apply-an-operation-to-every-character-in-a-string
\def\removeargs #1{\doremoveargs#1$\wholeString\unskip}
\def\doremoveargs#1#2\wholeString{\if#1$%
\else\if#1({()}\else{#1}\taketherest#2\fi\fi}
\def\taketherest#1\fi
{\fi \doremoveargs#1\wholeString}

% Note: still doesn't work properly when called on macro output ...
% i.e., \dottoslash{\concatnames{model}{base}{foo}} fails 
\def\dottoslash #1{\dodottoslash#1$\wholeString\unskip}
\def\dodottoslash#1#2\wholeString{\if#1$%
\else\if#1.{/}\else{#1}\fi\dottaketherest#2\fi}
\def\dottaketherest#1\fi{\fi \dodottoslash#1\wholeString}

\def\hashtodot #1{\dohashtodot#1$\wholeString\unskip}
\def\dohashtodot#1#2\wholeString{\if#1$X%
\else\if#1\#{.}\else{#1}\fi\hashtaketherest#2\fi}
\def\hashtaketherest#1\fi{\fi \dohashtodot#1\wholeString}

%\dollartodot{#1} does the same thing as \StrSubstitute[0]{#1}{\$}{.}
% from the packahe xstring. We define \dollartodot instead because
% LaTeXML does not implement xstring.
%
% Note that for the substituion to work, we need \ifx instead of \if,
% since otherwise escaped characters won't work properly:
% if #1 = \$, then \if#1* seems to compare '\' and '$' (and output '*'),
% rather than comparing '$' to '*'
\def\dollartodot #1{\dodollartodot#1*\wholeString\unskip}
\def\dodollartodot#1#2\wholeString{\ifx#1*%
\else \ifx#1\${.}\else{#1}\fi\dollartaketherest#2\fi}
\def\dollartaketherest#1\fi{\fi \dodollartodot#1\wholeString}

% concatenates up to three class/method names together, adding '.' characters
% between them. The first and/or second argument may be empty, in which case
% the '.' is omitted. To check to see if these arguments are empty, we
% use a contruction '\if#1@@', which will return true iff #1 is empty
% (on the assumption that #1 will not contain a '@' character).
\def\concatnames
#1#2#3{\if#1@@\if#2@@#3\else #2.#3\fi\else\if#2@@#1.#3\else#1.#2.#3\fi\fi}

\newcommand{\javabase}{}
\newcommand{\setjavabase}[1]{\renewcommand{\javabase}{#1}}

\def\artisynthDocBase{@ARTISYNTHDOCBASE}

\iflatexml
\def\ifempty#1{\def\temp{#1}\ifx\temp\empty}%
\newcommand{\artisynthManual}[3][]{%
   \ifempty{#1}
      \href{@ARTISYNTHDOCBASE/#2/#2.html}{#3}%
    \else
      \href{@ARTISYNTHDOCBASE/#1/#2.html}{#3}%
    \fi
}
\else
\newcommand{\artisynthManual}[3][]{%
\href{https://www.artisynth.org/@ARTISYNTHDOCBASE/#2.pdf}{#3}}
\fi

%\href{@ARTISYNTHDOCBASE/#2/#2.html}{#3}}



\newcommand{\javaclassx}[2][]{%
% Includes code to prevent an extra '.' at the front if #1 is empty. It
% works like this: if '#1' is empty, then '#1.' expands to '.', and so 
% '\if#1..' will return true, in which case we just output '#2'.
\href{@JDOCBEGIN/\concatnames{\javabase}{#1}{#2}@JDOCEND}{#2}}
\newcommand{\javaclass}[2][]{%
\href{@JDOCBEGIN/\concatnames{}{#1}{#2}@JDOCEND}{\dollartodot{#2}}}
\newcommand{\javaclassAlt}[2]{%
\href{@JDOCBEGIN/\concatnames{}{}{#1}@JDOCEND}{#2}}

\newcommand{\javamethodArgsx}[2][]{%
\href{@JDOCBEGIN/\concatnames{\javabase}{#1}{#2}@JDOCEND}{#2}}
\newcommand{\javamethodArgs}[2][]{%
\href{@JDOCBEGIN/\concatnames{}{#1}{#2}@JDOCEND}{#2}}
\newcommand{\javamethodAlt}[2]{%
\href{@JDOCBEGIN/\concatnames{}{}{#1}@JDOCEND}{#2}}
\newcommand{\javamethodAltx}[2]{%
\href{@JDOCBEGIN/\concatnames{\javabase}{}{#1}@JDOCEND}{#2}}

\newcommand{\javamethodNoArgsx}[2][]{%
\href{@JDOCBEGIN/\concatnames{\javabase}{#1}{#2}@JDOCEND}{\removeargs{#2}}}
\newcommand{\javamethodNoArgs}[2][]{%
\href{@JDOCBEGIN/\concatnames{}{#1}{#2}@JDOCEND}{\removeargs{#2}}}

\newcommand{\javamethod}{\@ifstar\javamethodNoArgs\javamethodArgs}
\newcommand{\javamethodx}{\@ifstar\javamethodNoArgsx\javamethodArgsx}

%%%%%%%%%%%%%%%%%%%%%%%%%%%%%%%%%%%%%%%%%%%%%%%%%%%%%%%%%%%%%%%%%%%%%%%%%%%%%
%
% Define macros for sidebars
%
%%%%%%%%%%%%%%%%%%%%%%%%%%%%%%%%%%%%%%%%%%%%%%%%%%%%%%%%%%%%%%%%%%%%%%%%%%%%%

\iflatexml
\newenvironment{sideblock}{\begin{quote}}{\end{quote}}
\else
\usepackage[strict]{changepage}
\definecolor{sidebarshade}{rgb}{1.0,0.97,0.8}
\newenvironment{sideblock}{%
    \def\FrameCommand{%
    \hspace{1pt}%
    {\color{sidebar}\vrule width 2pt}%
    %{\vrule width 2pt}%
    {\color{sidebarshade}\vrule width 4pt}%
    \colorbox{sidebarshade}%
  }%
  \MakeFramed{\advance\hsize-\width\FrameRestore}%
  \noindent\hspace{-4.55pt}% disable indenting first paragraph
  \begin{adjustwidth}{}{7pt}%
  %\vspace{2pt}\vspace{2pt}%
}
{%
  \vspace{2pt}\end{adjustwidth}\endMakeFramed%
}
\fi

\iflatexml
\newenvironment{shadedregion}{%
  \definecolor{shadecolor}{rgb}{0.96,0.96,0.98}%
  \begin{shaded*}%
% Put text inside a quote to create a surrounding blockquote that
% will properly accept the color and padding attributes
  \begin{quote}%
}
{%
  \end{quote}%
  \end{shaded*}%
}
\else
\newenvironment{shadedregion}{%
  \definecolor{shadecolor}{rgb}{0.96,0.96,0.98}%
  \begin{shaded*}%
}
{%
  \end{shaded*}%
}
\fi

% Wanted to create a 'listing' environment because lstlisting is
% tedious to type and because under latexml it may need
% some massaging to get it to work properly. But hard to do
% because of the verbatim nature of listing
%\iflatexml
%\newenvironment{listing}{\begin{lstlisting}}{\end{lstlisting}}%
%\else
%\newenvironment{listing}{\begin{lstlisting}}{\end{lstlisting}}%
%\fi

\iflatexml\else
% fancyhdr was complaining that it wanted a 36pt header height ...
\setlength{\headheight}{36pt}
\fi

% macro for backslash character
\newcommand\BKS{\textbackslash}

% macro for double hyphen (to prevent conversion of -- into -)
\newcommand\DHY{-{}-}

% Convenience stuff
\newcommand{\ifLaTeXMLelse}[2]{%
  \iflatexml %
  #1 %
  \else %
  #2 %
  \fi %
}

\newcommand{\ifLaTeXML}[1]{ %
  \iflatexml %
  #1 %
  \fi %
}

% new methodtable environment for documenting methods

% base width of the method table
\newlength{\methodtablewidth}
\iflatexml
\setlength{\methodtablewidth}{1.6\textwidth}
\else
\setlength{\methodtablewidth}{0.94\textwidth}
\fi
% horizontal space added at end of call to \methodentry
\newlength{\methodskip}
\setlength{\methodskip}{0pt}
% lengths set inside methodtable environment:
\newlength{\methodsiglength} % length of the method signature
\newlength{\methodcomlength} % length of the method comment
\setlength{\methodsiglength}{0.5\methodtablewidth}
\setlength{\methodcomlength}{0.5\methodtablewidth}

% command to add a method to a method table:
% arg #1: package and signature for finding URL
% arg #2: anchor text
% arg #3: comment describing the method
\newcommand{\methodentry}[3]{%
\javamethodAlt{#1}{\parbox[t]{\methodsiglength}{#2}}&
{\parbox[t]{\methodcomlength}{#3}}\\%
\noalign{\vspace{\methodskip}}}

% methodtable environment takes two arguments, both scale factors for
% methodtablewidth:
% arg #1: width of the method signature column
% arg #2: width of the method comment column
\newenvironment{methodtable}[3][0pt]{%
\begingroup
\setlength{\topskip}{0pt}
\setlength{\methodskip}{#1}
\setlength{\methodsiglength}{#2\methodtablewidth}%
\setlength{\methodcomlength}{#3\methodtablewidth}%
\iflatexml
\begin{snugshade}
\else
\begin{tcolorbox}
\fi
\renewcommand{\arraystretch}{1}
\begin{tabular}{ll}}{%
\end{tabular}
\renewcommand{\arraystretch}{1}
\iflatexml
\end{snugshade}
\else
\end{tcolorbox}
\fi
\endgroup}

% commands for added top, mid and bottom lines in the table.
% uses booktabs for PDF, regular hline for HTML
\newcommand{\topline}{\iflatexml\hline\else\toprule\fi}
\newcommand{\midline}{\iflatexml\hline\else\midrule\fi}
\newcommand{\botline}{\iflatexml\hline\else\bottomrule\fi}
\newcommand{\blankline}{%
\multicolumn{2}{l}{\iflatexml{@SPACE}\else\phantom{M}\fi}\\}%
% add vertical space within a two colum method environment
\newcommand{\methodspace}[1]{%
\iflatexml
\multicolumn{2}{l}{@VERTSPACE[#1]}\\
\else
\noalign{\vspace{#1}}%
\fi}%
% break a line and add an indentation of 1em
\newcommand{\brh}{\\\phantom{M}}

\makeatother

\def\matl{\left(\begin{matrix}}
\def\matr{\end{matrix}\right)}

\def\Bthe{\boldsymbol\theta}
\def\Btau{\boldsymbol\tau}
\def\Bom{\boldsymbol\omega}
\def\Bdel{\boldsymbol\delta}
\def\Blam{\boldsymbol\lambda}
\def\Bphi{\boldsymbol\phi}
\def\Bxi{\boldsymbol\xi}
\def\Bgam{\boldsymbol\gamma}
\def\Bsig{\boldsymbol\sigma}
\def\Bnu{\boldsymbol\nu}
\def\Bmu{\boldsymbol\mu}

\def\A{{\bf A}}
\def\B{{\bf B}}
\def\C{{\bf C}}
\def\D{{\bf D}}
\def\F{{\bf F}}
\def\G{{\bf G}}
\def\H{{\bf H}}
\def\I{{\bf I}}
\def\J{{\bf J}}
\def\K{{\bf K}}
\def\Jc{{\bf J}_c}
\def\L{{\bf L}}
\def\M{{\bf M}}
\def\N{{\bf N}}
\def\O{{\bf O}}
\def\P{{\bf P}}
\def\Q{{\bf Q}}
\def\R{{\bf R}}
\def\T{{\bf T}}
\def\U{{\bf U}}
\def\W{{\bf W}}
\def\X{{\bf X}}
\def\Minv{{\bf M}^{-1}}

\def\a{{\bf a}}
\def\b{{\bf b}}
\def\c{{\bf c}}
\def\d{{\bf d}}
\def\e{{\bf e}}
\def\f{{\bf f}}
\def\g{{\bf g}}
\def\k{{\bf k}}
\def\l{{\bf l}}
\def\m{{\bf m}}
\def\n{{\bf n}}
\def\p{{\bf p}}
\def\q{{\bf q}}
\def\r{{\bf r}}
\def\u{{\bf u}}
\def\v{{\bf v}}
\def\w{{\bf w}}
\def\x{{\bf x}}
\def\y{{\bf y}}
\def\z{{\bf z}}

\def\ma{{\bf m}_\alpha}
\def\mb{{\bf m}_\beta}
\def\va{{\bf v}_\alpha}
\def\vb{{\bf v}_\beta}
\def\vp{{\bf v}_\rho}
\def\vk{{\bf v}_k}
\def\ua{{\bf u}_\alpha}
\def\ub{{\bf u}_\beta}
\def\uk{{\bf u}_k}
\def\uj{{\bf u}_j}
\def\mar{{\bf m}_{\alpha r}}
\def\mbr{{\bf m}_{\beta r}}

\def\Maa{{\bf M}_{\alpha\alpha}}
\def\Mab{{\bf M}_{\alpha\beta}}
\def\Mba{{\bf M}_{\beta\alpha}}
\def\Mbb{{\bf M}_{\beta\beta}}
\def\hatMaa{\hat{\bf M}_{\alpha\alpha}}
\def\hatMab{\hat{\bf M}_{\alpha\beta}}
\def\hatMba{\hat{\bf M}_{\beta\alpha}}
\def\hatMbb{\hat{\bf M}_{\beta\beta}}
\def\Mbp{{\bf M}_{\beta\rho}}
\def\Map{{\bf M}_{\alpha\rho}}
\def\Mpa{{\bf M}_{\rho\alpha}}
\def\Mpb{{\bf M}_{\rho\beta}}
\def\Mpp{{\bf M}_{\rho\rho}}
\def\Mbk{{\bf M}_{\beta k}}
\def\Mak{{\bf M}_{\alpha k}}
\def\Mka{{\bf M}_{k\alpha}}
\def\Mkb{{\bf M}_{k\beta}}
\def\Mkk{{\bf M}_{kk}}

\def\Ga{{\bf G}_{\alpha}}
\def\Gp{{\bf G}_{\rho}}
\def\Gaa{{\bf G}_{\alpha\alpha}}
\def\Gab{{\bf G}_{\alpha\beta}}
\def\Gba{{\bf G}_{\beta\alpha}}
\def\Gbb{{\bf G}_{\beta\beta}}
\def\Gap{{\bf G}_{\alpha\rho}}
\def\Gpa{{\bf G}_{\rho\alpha}}
\def\Gbp{{\bf G}_{\beta\rho}}
\def\Gak{{\bf G}_{\alpha k}}
\def\Gka{{\bf G}_{k\alpha}}
\def\Gja{{\bf G}_{j\alpha}}
\def\Gkb{{\bf G}_{k\beta}}
\def\Gbk{{\bf G}_{\beta k}}

\def\lama{\Blam_{\alpha}}
\def\lamb{\Blam_{\beta}}
\def\lamp{\Blam_{\rho}}
\def\lamk{\Blam_{k}}
\def\lams{\Blam_{\sigma}}

\def\ba{{\bf b}_{\alpha}}
\def\bb{{\bf b}_{\beta}}
\def\fp{{\bf f}_{\rho}}
\def\fa{{\bf f}_{\alpha}}
\def\qa{{\bf q}_{\alpha}}
\def\qb{{\bf q}_{\beta}}
\def\za{{\bf z}_{\alpha}}
\def\zb{{\bf z}_{\beta}}
\def\wa{{\bf w}_{\alpha}}
\def\wb{{\bf w}_{\beta}}

\def\Na{\bar{\bf N}_{\alpha}}
\def\Nb{\bar{\bf N}_{\beta}}

\def\Up{{\bf U}_p}
\def\Un{{\bf U}_n}

\def\dFdl{\frac{\partial F}{\partial l}}
\def\dFddl{\frac{\partial F}{\partial \dot l}}

\def\Sr{s_\theta}
\def\Cr{c_\theta}
\def\Sp{s_\phi}
\def\Cp{c_\phi}
\def\Sy{s_\psi}
\def\Cy{c_\psi}
\def\Sa{s_{\alpha}}
\def\Ca{c_{\alpha}}
\def\Vp{v_{\phi}}


\def\mech{artisynth.core.mechmodels}
\def\rend{artisynth.core.renderables}
\def\fem{artisynth.core.femmodels}
\def\mgeo{maspack.geometry}

\setcounter{tocdepth}{5}
\setcounter{secnumdepth}{3}

\title{Creating Models in ArtiSynth}
\author{John Lloyd}
\setpubdate{October 2018}

\iflatexml
\date{}
\fi

% graphics paths
\graphicspath{{./}{images/}}

\begin{document}

%\frontmatter

%\layout
\maketitle

%\iflatexml{\large\pubdate}\fi

%\tableofcontents

% basic links to other docs: http://www.artisynth.org/doc/html/xxx/xxx.html#sec

\section{Introduction}

This note provides a broad overview of
how to create models in ArtiSynth. It begins with a general overview
of the different modeling components, and then discusses the typical steps
required to assemble (in code) these components into a working model.

\section{Component Overview}

In a very broad sense, ArtiSynth components divide into (a) dynamic
components that carry the state variables (e.g., positions and
velocities) associated with the simulation, (b) connecting components
that provide forces and constraints which control how the dynamic
components move and interact, and finally (c) passive components such
as meshes or renderable objects.

Some of this component functionality may be exhibited by composite
components that contains multiple sub-components.  A good example of
this is a finite element model (\javaclass[\fem]{FemModel3d}), which
contains nodes, elements, meshes, and other possible
constituents. Each of the nodes represents a dynamic component.
However, the model as a whole acts as both a force effector (applying
forces to the nodes), and also as a constrainer (applying hard
incompressibility to the nodal positions when requested).

\subsection{Dynamic Components}
\label{DynamicComponents:sec}

Dynamic components carry the position and velocity states that
collectively describe the model's configuration. Generally, dynamic
components move in response to forces acting on them, but components
can also be made "parametric", so that their positions and velocities
are controlled directly by external inputs. Some components, such as
points and frames, are {\it only} parametric, and their state can be
changed only by external inputs or by being attached to other dynamic
components. Dynamic components include:

\begin{itemize}

\item Points, implemented by \javaclass[\mech]{Point}, describing a 3
DOF point in space (parametric only).

\item Frames, implemented by \javaclass[\mech]{Frame}, describing a 6 DOF
spatial coordinate frame (parametric only).

\item Particles, which are points with mass and are implemented
by \javaclass[\mech]{Particle}, which is a subclass of \javaclass[\mech]{Point}.

\item Rigid bodies, which are frames with inertia and are implemented
by \javaclass[\mech]{RigidBody}, which is a subclass of \javaclass[\mech]{Frame}.

\item Finite element nodes, implemented by \javaclass[\fem]{FemNode3d}
which is a subclass of \javaclass[\mech]{Particle}.

\item Markers, which are points attached to other components, using
the attachment mechanism described in Section \ref{attachments:sec}.
They are implemented as subclasses of \javaclass[\mech]{Point}, and include
\javaclass[\mech]{FrameMarker} and \javaclass[\fem]{FemMarker}, which
can be attached to frames or finite element models, respectively.

\end{itemize}

There is also support for "deformable" bodies, that are like rigid
bodies with some extra DOFs to handle deformation, but these are
currently experimental and aren't used much.

\subsection{Force Effectors}

These generate the forces that act on dynamic components, and include:

\begin{itemize}

\item Axial springs (class \javaclass[\mech]{AxialSpring}), which a applies
a force between two points.

\item Multi-point springs (class \javaclass[\mech]{MultiPointSpring}),
which applies forces between multiple points connected together in a
series of segments. Multi-point springs can also {\it wrap} around
rigid obstacles.

\item Frame springs (class \javaclass[\mech]{FrameSpring}), which apply
translational and rotational forces between two rigid bodies.

\item Point-to-point muscles (class \javaclass[\mech]{Muscle}) which are
axial springs that also incorporate an intrinsic tension based on an
"excitation" input.

\item Multi-point muscles (class \javaclass[\mech]{MultiPointMuscle}), which are
multi-point springs that also incorporate an intrinsic tension based on
an "excitation" input.

\item Finite element models (main class
\javaclass[\fem]{FemModel3d}), which apply forces to their nodes,
based on the local deformations of their elements and (for damping)
their nodal velocities.

\item Dynamic components can act as their own force effectors by
providing velocity base damping forces

\item Gravity loading, implemented directly by either finite element
model or within \javaclass[\mech]{MechModel}, using the {\tt applyGravity()}
method supplied by dynamic components.

\end{itemize}

\subsection{Constrainers}

Constrainers restrict the ways in which dynamic components can move.
Constraint relationships involve either qualities (bilateral) or
inequalities (unilateral). Bilateral constraints include:

\begin{itemize}

\item Joints, such as \javaclass[\mech]{RevoluteJoint},
\javaclass[\mech]{RollPitchJoint}, \javaclass[\mech]{SphericalJoint} and
\javaclass[\mech]{SolidJoint}, which constrain the relative motions of
two frames or rigid bodies.

\item RigidBody to planar constraints, such as
\javaclass[\mech]{PlanarConnector} and
\javaclass[\mech]{SegmentedPlanarConnector}, which constrain a point
on a rigid body to lie on a plane or segmented plane.

\item Particle constraints, such as
\javaclass[\mech]{ParticlePlaneConstraint} and
\javaclass[\mech]{ParticleMeshConstraint}, which constrain a particle
to lie on a plane or a mesh.

\item Hard incompressibility within finite element models.

\end{itemize}

Unilateral constraints include:

\begin{itemize}

\item Contact between rigid bodies or finite elements models. The
actual contact constraints are generated by the
\javaclass[\mech]{CollisionManager} component of
\javaclass[\mech]{MechModel}.

\item Joint limits in joints such as \javaclass[\mech]{RevoluteJoint},
\javaclass[\mech]{RollPitchJoint}, and \javaclass[\mech]{SphericalJoint}.

\end{itemize}

\subsection{Attachments}
\label{attachments:sec}

Attachments directly connect a "slave" dynamic component to one or
more "master" components. Attachments are really just constraints, but
they're implemented differently: attachments involve an equality
relationship between the velocity of the slave and the velocities of
the masters, which can be used to remove the slave from the physical
system that has to be solved. This in turn reduces the size of the
physical system and can increase solution times.

At present, attachments are used to connect points and frames to other
dynamic components.

\begin{itemize}

\item Point attachments, including
\javaclass[\mech]{PointParticleAttachment},
\javaclass[\mech]{PointFrameAttachment} and
\javaclass[\fem]{PointFem3dAttachment}, attach a point to either a
particle, a frame, or a finite element model. Another attachment,
\javaclass[\fem]{PointSkinAttachment}, can attach a point to a
\javaclass[\fem]{SkinMeshBody} (see below).

\item Frame attachments, including
\javaclass[\mech]{FrameFrameAttachment} and
\javaclass[\fem]{FrameFem3dAttachment}, can attach a frame to either
another frame or a finite element model.

\end{itemize}

Markers, as described in Section \ref{DynamicComponents:sec}, work by
using their own internal attachments.

\subsection{Mesh geometry}

Models often include various kind of mesh geometry, often for
visualization but also to control various properties of the
simulation.

The {\tt maspack.geometry} package supports three types of mesh:
\javaclass[\mgeo]{PolygonalMesh}, \javaclass[\mgeo]{PolylineMesh}, and
\javaclass[\mgeo]{PointMesh}, which describe meshes composed of
polygons (usually triangles), line segments, and points,
respectively. Each of these may be added to a model or incorporated
into one of it's components. Meshes can be purely passive, playing no
role in the simulation, or they may be attached to underlying dynamic
components in some way.

Within ArtiSynth, meshes are contained within a \javaclass[\mech]{MeshComponent},
for which the following subclasses exist:

\begin{itemize}

\item \javaclass[\mech]{FixedMeshBody}, which contains a passive mesh used for
visualization purposes. The mesh can be positioned in space using a
rigid transform (its {\it pose}) but otherwise is generally fixed.

\item \javaclass[\fem]{FemMeshComp} contains a deformable mesh which is embedded
within a finite element model, and hence moves with that model. Such
meshes are used to implement the model's surface mesh, or various
submeshes within the model. Line segment meshes can be used to
indicate the directions of fibre fields within an muscle activated FEM,
while triangular meshes can be used for collision response between the
FEM and other objects, or to display ``heat maps'' of stress or strain
values. All FEM embedded meshes are implemented by having their
vertices ``attached'' to one or more FEM nodes, in a manner identical
to that used for FEM markers.

\item \javaclass[\mech]{RigidMeshComp} contains a rigid mesh that is
associated with a \javaclass[\mech]{RigidCompositeBody}. The mesh
moves in space as the rigid body moves, but otherwise does not
deform. Rigid body meshes can be used for collision response with
other FEMs or rigid bodies, and can also be used to determine the
inertia properties of the body given an assumption of uniform density.

\begin{sideblock}
At present, the use of {\tt RigidMeshComp} is presently restricted to
{\tt RigidCompositeBody}, which is a rigid body that may contain
multiple meshes. The standard {\tt RigidBody} component at present may
have only one mesh, which it contains directly. Future plans call
for merging the multiple mesh feature of {\tt RigidCompositeBody} 
into {\tt RigidBody}.
\end{sideblock}

\item \javaclass[\fem]{SkinMeshBody} contains a deformable {\it skin}
mesh that is attached to a combination of rigid bodies and finite
element models and deforms in response to their motions. Skin meshes
are used to implement a uniform covering over a variety of other model
components.

\end{itemize}

\subsection{Renderables}

Some components exist purely for visualization purposes. Many of these
are contained within {\tt artisynth.core.renderables}, and include:

\begin{itemize}

\item \javaclass[\rend]{ColorBar}, used to display a color bar,
typically as a key for a heat map.

\item \javaclass[\rend]{DicomViewer}, used to display medical imagery
from DICOM files.

\item \javaclass[\rend]{TextComponent2d} and
\javaclass[\rend]{TextComponent3d}, used for displaying text in 2D or
3D.

\item \javaclass[\rend]{LightComponent}, for adding illumination to
the scene.

\end{itemize}

\section{Assembling Components}

A common workflow for assembling an ArtiSynth model is the following:

\begin{itemize}

\item {\bf Creating the primary components}, such as finite element
models, rigid bodies, meshes, and isolated particles. This generally
entails defining the component geometry (either by reading it in from
an external file or calling a suitable factory method), transforming
the geometry if necessary (through translation, rotation, scaling, or
applying a more general deformation), and setting material parameters
such as density, stiffness coefficients, and damping.

\item {\bf Adding the connecting components}, such as springs,
muscles, joints, contact constraints, and attachments. This typically
requires identifying the locations of the points and frames that are
used to position or anchor these components. Such locations may be
known a-priori, and either read from a file or simply defined in code,
or they may be determined computationally.

\item {\bf Adding the controls and monitors}, such as control panels,
input and output probes, controllers, and monitors that are used to
control and monitor the simulation. Control panels and probes are
usually built by connecting widgets and input/output channels directly
to the {\it properties} of selected components, wheres controllers and
monitors can be implemented with application-defined code (and so can
perform arbitrary tasks).

\end{itemize}

\end{document}
