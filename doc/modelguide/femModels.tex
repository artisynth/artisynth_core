\section{Finite Element Models}
\label{FEMModels:sec}

This section details how to construct three-dimensional finite element models,
and how to couple them with the other simulation components described in
previous sections (e.g.~particles and rigid bodies).  Finite element
\emph{muscles}, which have additional properties that allow them to contract
given activation signals, are discussed in Section \ref{sec:fem:muscle}.

\subsection{Overview}
\label{sec:fem:overview}

The finite element method (FEM) is a numerical technique used for solving a 
system of partial differential equations (PDEs) over some domain.  The general
approach is to divide the domain into a set of building blocks, referred to
as \emph{elements}, over which the problem can be discretized.  Using this
discretization, the differential system is converted into an algebraic one,
which is often linearized and iteratively solved.

In ArtiSynth, the PDEs considered are the governing equations of
continuum mechanics: the conservation of mass, momentum and energy.  To 
complete the system, a \emph{constitutive equation} is required that describes
the stress-strain response of the material.  This constitutive equation is what
disguishes between material types.  The domain of the system is the 3D spatial 
domain.  This must be divided into small elements which accurately represent
the geometry. Within each element, the PDEs are
sampled at a set of points, referred to as \emph{integration points}, and 
terms are numerically integrated to form an algebraic system to solve.

The purpose of the rest of this section is to describe the construction and
use of finite elements models within ArtiSynth.  It does not discuss the
mathematical framework or theory.
For an in-depth coverage of the nonlinear finite element method, as applied
to continuum mechanics, the reader is referred to the textbook by Bonet and 
Wood \cite{bonet:fem:2000}.

\subsubsection{Structure}
\label{sec:fem:structure}

The basic type of finite element model is implemented in the class 
\javaclass[artisynth.core.femmodels]{FemModel3d}.  Each FEM model contains
at least three lists of sub-components:

\begin{description}
\item[Elements]\mbox{}

The building blocks of the model.  These define the sub-units over which
the system is numerically integrated.

\item[Nodes]\mbox{}

The particle-like dynamic components of the model.  These lie at the corners
of the elements, and carry all the mass.

\item[Meshes]\mbox{}

The geometry in the model.  This will include the surface mesh, and any other
embedded geometries which are used for either display purposes, or for physical
interactions (e.g.~collision).
\end{description}

Additionally, the model itself has several important properties which should be
set:
\begin{center}
	\begin{tabular}{|ll|}
		\hline
		Property & Description\\
		\hline
		{\tt density} & The density of the model\\
		{\tt particleDamping} & Proportional damping associated with the 
		    particle-like motion of the FEM nodes.\\
		{\tt stiffnessDamping} & Proportional damping associated with the 
		    system's stiffness term.\\
		{\tt material} & An object that describes the material's 
		    \emph{constitutive law} (i.e.~it's 
stress-strain relationship).\\
		\hline
	\end{tabular}
\end{center}

\begin{description}
\item[density] \mbox{}

The density of the model

\item[particleDamping] \mbox{}

Proportional damping associated with the particle-like motion of the FEM nodes.

\item[stiffnessDamping]\mbox{}

Proportional damping associated with the system's stiffness term.

\item[material]\mbox{}

An object that describes the material's \emph{constitutive law} (i.e.~it's 
stress-strain relationship).
\end{description}


\subsubsection{Materials}
\label{sec:fem:materials}

\subsubsection{Boundary conditions}

\subsection{FEM model creation}

\subsubsection{Using factory methods}

\subsubsection{Using external meshes}

\subsubsection{Using direct code}

\subsubsection{A simple beam model}

\begin{figure}[ht]
\begin{center}
\iflatexml
 \includegraphics[]{images/FemBeam}
\else
 \includegraphics[width=3.75in]{images/FemBeam}
\fi
\end{center}
\caption{FemBeam model loaded into ArtiSynth.}
\label{FemBeam:fig}
\end{figure}

\subsection{FEM Geometry}

\subsubsection{Surface meshes}

\subsubsection{Embedding geometry within an FEM}

\subsubsection{A beam with an embedded sphere}

% EmbeddedFemSphere

\subsection{Node attachments}
\label{NodeAttachments:sec}

\subsubsection{General principles}

\subsubsection{Connecting a beam to a block}

\begin{figure}[ht]
\begin{center}
\iflatexml
 \includegraphics[]{images/FemBeamWithBlock}
\else
 \includegraphics[width=3.75in]{images/FemBeamWithBlock}
\fi
\end{center}
\caption{FemBeamWithBlock model loaded into ArtiSynth.}
\label{FemBeamWithBlock:fig}
\end{figure}

% FemBeamWithBlock

\subsubsection{Connecting two FEMs together}

% ConnectedFems

\subsection{FEM markers}

\subsubsection{Embedding particles in FEMs}

\subsubsection{Attaching a FEM beam to a muscle}

\begin{figure}[ht]
\begin{center}
\iflatexml
 \includegraphics[]{images/FemBeamWithMuscle}
\else
 \includegraphics[width=3.75in]{images/FemBeamWithMuscle}
\fi
\end{center}
\caption{FemBeamWithMuscle model loaded into ArtiSynth.}
\label{FemBeamWithMuscle:fig}
\end{figure}

% FemBeamWithMuscle

\subsection{Muscle activated FEM models}
\label{sec:fem:muscle}

\subsubsection{FemMuscleModel}

\subsubsection{Activation with fibres}

\subsubsection{Activation with embedded materials}

\subsubsection{Comparision with two beam examples}

\begin{figure}[ht]
\begin{center}
\iflatexml
 \includegraphics[]{images/FemMuscleBeams}
\else
 \includegraphics[width=3.75in]{images/FemMuscleBeams}
\fi
\end{center}
\caption{FemMuscleBeams model loaded into ArtiSynth.}
\label{FemMuscleBeams:fig}
\end{figure}

% FemMuscleBeams

\subsection{Collisions}

\subsubsection{Colliding with FEM geometry}

\subsubsection{Colliding with the surface mesh}

\begin{figure}[ht]
\begin{center}
\iflatexml
 \includegraphics[]{images/FemMuscleHeart}
\else
 \includegraphics[width=3.75in]{images/FemMuscleHeart}
\fi
\end{center}
\caption{FemMuscleHeart model loaded into ArtiSynth.}
\label{FemMuscleHeart:fig}
\end{figure}

% FemMuscleHeart

\subsubsection{Colliding with an embedded sphere}

\begin{figure}[ht]
\begin{center}
\iflatexml
 \includegraphics[]{images/FemCollisions}
\else
 \includegraphics[width=3.75in]{images/FemCollisions}
\fi
\end{center}
\caption{FemCollisions model loaded into ArtiSynth.}
\label{FemCollisions:fig}
\end{figure}

\subsection{Visualization}

stuff stuff stuff

\subsubsection{Rendering settings}

stuff stuff stuff

\subsubsection{Stress and strain plotting}

stuff stuff stuff
