% Scale latexml image sizes 2x
\newlength{\imglength}
\newcommand{\setlengthLaTeXML}[3]{ %
	\iflatexml %
	\setlength{#1}{#2} %
	\else %
	\setlength{#1}{#3} %
	\fi %
}

\section{Finite Element Models}
\label{FEMModels:sec}

This section details how to construct three-dimensional finite element models,
and how to couple them with the other simulation components described in
previous sections (e.g.~particles and rigid bodies).  Finite element
\emph{muscles}, which have additional properties that allow them to contract
given activation signals, are discussed in Section \ref{sec:fem:muscle}.  An 
example FEM model of the masseter, coupled to a rigid jaw and maxilla, is shown 
in Figure \ref{fig:fem:masseter}.

\begin{figure}[ht]
	\centering
	\setlengthLaTeXML{\imglength}{0.8\textwidth}{0.6\textwidth}
	\includegraphics[width=\imglength]{images/fem_masseter}
	\caption{Finite element model of the masseter, coupled to the jaw and 
	         maxilla. \label{fig:fem:masseter}} 
\end{figure}

\subsection{Overview}
\label{sec:fem:overview}

The finite element method (FEM) is a numerical technique used for solving a 
system of partial differential equations (PDEs) over some domain.  The general
approach is to divide the domain into a set of building blocks, referred to
as \emph{elements}.  These partition the space, and form local domains over
which the system of equations can be locally approximated. The corners of these
elements, the \emph{nodes}, become control points in a discretized system.  
The solution is then assumed to be
smoothly interpolated across the elements based on values determined at the
nodes.  Using this discretization, the differential system is converted into 
an algebraic one, which is often linearized and iteratively solved.

In ArtiSynth, the PDEs considered are the governing equations of
continuum mechanics: the conservation of mass, momentum and energy.  To 
complete the system, a \emph{constitutive equation} is required that describes
the stress-strain response of the material.  This constitutive equation is what
disguishes between material types.  The domain is the three-dimensional space
that the model occupies. This must be divided into small elements which 
accurately represent the geometry. Within each element, the PDEs are
sampled at a set of points, referred to as \emph{integration points}, and 
terms are numerically integrated to form an algebraic system to solve.

The purpose of the rest of this section is to describe the construction and
use of finite elements models within ArtiSynth.  It does not further discuss 
the mathematical framework or theory.
For an in-depth coverage of the nonlinear finite element method, as applied
to continuum mechanics, the reader is referred to the textbook by Bonet and 
Wood \cite{bonet:fem:2000}.

\subsubsection{FemModel3d}
\label{sec:fem:structure}

The basic type of finite element model is implemented in the class 
\javaclass[artisynth.core.femmodels]{FemModel3d}.  This class controls some
properties that are used by the model as a whole.  The key ones that affect
simulation dynamics are:
\begin{center}
	\begin{tabular}{|ll|}
		\hline
		Property & Description\\
		\hline
		{\tt density} & The density of the model\\
		{\tt material} & An object that describes the material's 
		    \emph{constitutive law} (i.e.~its stress-strain relationship).\\
		{\tt particleDamping} & Proportional damping associated with the 
		    particle-like motion of the FEM nodes.\\
		{\tt stiffnessDamping} & Proportional damping associated with the 
		    system's stiffness term.\\
		\hline
	\end{tabular}
\end{center}
Keep in mind that ArtiSynth is essentially ``unitless'' (Section 
\ref{sec:mechii:units}), so it is the responsibility of the developer to
ensure that all properties are specified in a compatible way.  

The density of the model is used to compute the mass distribution throughout
the model.  Note that we use a \emph{diagonally lumped mass matrix} (DLMM)
formulation, so the mass is assumed to be concentrated at the location of
the discritized FEM nodes.  To allow for a spatially-varying density,
a mass can later be specified for each node individually.

The FEM's {\tt material} is a delegate object used to compute stress and 
stiffness within individual elements.  It handles the \emph{constitutive}
component of the model.  Materials will be discussed in more detail in
Section \ref{sec:fem:materials}.

The two damping parameters are related to \emph{Rayleigh damping}, which
is used to dissipate energy within finite element models.  There are two 
proportional damping terms: one related to the system's mass, and one related 
to stiffness.  The resulting damping force applied is
\begin{align}
	\f_d & = - (d_M \M + d_K\K)\v,
\end{align}
where $d_M$ is the value of {\tt particleDamping}, $d_K$ is the value of 
{\tt stiffnessDamping}, $\M$ is the FEM model's lumped mass matrix, $\K$ is 
the FEM's stiffness matrix, and $\v$ is the concatenated vector of FEM node
velocities.  Since the lumped mass matrix is diagonal, the mass-related
component of damping can be applied separately to each FEM node.  Thus, the
mass component reduces to the same system as Equation \eqref{eqn:pointdamping},
which is why it is referred to as ``particle damping''.

\subsubsection{Component Structure}

Each \javaclass[artisynth.core.femmodels]{FemModel3d} contains three 
lists of sub-components:

\begin{description}
\item[Nodes]\mbox{}

The particle-like dynamic components of the model.  These lie at the corners
of the elements and carry all the mass (due to DLMM formulation).

\item[Elements]\mbox{}

The spatial building blocks of the model.  These define the sub-units over 
which the system is numerically integrated.

\item[Meshes]\mbox{}

The geometry in the model.  This includes the surface mesh, and any other
embedded geometries.
\end{description}

An example showing each of these components is shown in Figure \ref{fig:fem}.


\begin{figure}[ht]
	\centering
	%\subfigure[][FEM model \label{fig:fem:model}] {\includegraphics[width=0.2\textwidth]{images/fem_embedded.png}}
	%\subfigure[][Nodes \label{fig:fem:nodes}] {\includegraphics[width=0.2\textwidth]{images/fem_embedded_nodes.png}}
	%\subfigure[][Elements \label{fig:fem:elements}] {\includegraphics[width=0.2\textwidth]{images/fem_embedded_elements.png}}
	%\subfigure[][Geometry \label{fig:fem:geometry}] {\includegraphics[width=0.2\textwidth]{images/fem_embedded_geometry.png}}
	\setlengthLaTeXML{\imglength}{1.5in}{1.2in}
	\begin{tabular}{cccc}
	\includegraphics[width=\imglength]{images/fem_embedded.png} & 
	\includegraphics[width=\imglength]{images/fem_embedded_nodes.png} &
	\includegraphics[width=\imglength]{images/fem_embedded_elements.png} &
	\includegraphics[width=\imglength]{images/fem_embedded_geometry.png}\\
	(a) FEM model & (b) Nodes & (c) Elements & (d) Geometry
	\end{tabular}
	\caption{Sub-components of \javaclass[artisynth.core.femmodels]{FemModel3d}. \label{fig:fem}}
\end{figure}

\paragraph{Nodes}
\ifLaTeXML{\newline}

The finite element nodes are implemented in the class 
\javaclass[artisynth.core.femmodels]{FemNode3d}, which is a subclass of 
\javaclass[artisynth.core.mechmodels]{Particle} (Section 
\ref{ParticlesAndSprings:sec}).  They are the main dynamic components of
the finite element model.  The key properties affecting simulation dynamics
are:
\begin{center}
	\begin{tabular}{|ll|}
		\hline
		Property & Description\\
		\hline
		{\tt restPosition} & The initial position of the node.\\
		{\tt position} & The current position of the node.\\
		{\tt velocity} & The current velocity of the node.\\
		{\tt mass} & The mass of the node.\\
		{\tt dynamic} & Whether the node is considered dynamic or parametric 
		                (e.g.~boundary condition).\\
		\hline
	\end{tabular}
\end{center}
The {\tt restPosition} property defines the node's position in the FEM model's 
``natural'' or ``undeformed'' state.  Rest positions are used to compute
an initial configuration for the model, from which strains are determined.  A
node's rest position can be updated in code using the method:
\javamethod[artisynth.core.femmodels]{FemNode3d.setRestPosition(Point3d)}.

\begin{sideblock}
If any node's rest positions are changed, the current values 
for stress and stiffness will become invalid.  They can be manually
updated using the method \javamethod[artisynth.core.femmodels] %
{FemModel3d.updateStressAndStiffness()} for the parent model. Otherwise,
stress and stiffness will be automatically updated at the beginning of the 
next time step. 
\end{sideblock}

The properties {\tt position} and {\tt velocity} give the node's current
3D state.  These are common to all point-like particles, as is the 
{\tt mass} property.  Here, however, {\tt mass} represents the lumped mass
of the immediately surrounding material.  Its value is initialized by equally
dividing mass contributions from each adjacent element, given their
densities.  For a finer control of spatially-varying densities,
node masses can be set manually after FEM creation.

The FEM node's {\tt dynamic} property specifies whether or not the 
node is considered when computing the dynamics of the system.  If not,
it is treated as being parametrically controlled.  This has implications
when setting boundary conditions (Section \ref{sec:fem:boundary}).

\paragraph{Elements}
\ifLaTeXML{\newline}

Elements are classified by their shape.  ArtiSynth supports the following 
element types:
\begin{center}
	\setlengthLaTeXML{\imglength}{1.5in}{1in}
	\begin{tabular}{c@{\hspace{5ex}}c@{\hspace{5ex}}c@{\hspace{5ex}}c}
		\includegraphics[height=\imglength]{images/fem_element_tet} &
		\includegraphics[height=\imglength]{images/fem_element_pyramid} &
		\includegraphics[height=\imglength]{images/fem_element_wedge} &
		\includegraphics[height=\imglength]{images/fem_element_hex} \\
		\javaclass[artisynth.core.femmodels]{TetElement} &
		\javaclass[artisynth.core.femmodels]{PyramidElement} & 
		\javaclass[artisynth.core.femmodels]{WedgeElement} &
		\javaclass[artisynth.core.femmodels]{HexElement}
	\end{tabular}
\end{center}

\begin{center}
	\begin{tabular}{|ll|}
		\hline
		Property & Description\\
		\hline
		{\tt material} & An object that describes the \emph{constitutive law} 
		                 within the element (i.e.~its stress-strain 
		                 relationship).\\
		\hline
	\end{tabular}
\end{center}

\paragraph{Geometry}
\ifLaTeXML{\newline}

\subsubsection{Materials}
\label{sec:fem:materials}

\subsubsection{Boundary conditions}
\label{sec:fem:boundary}

\subsection{FEM model creation}

\subsubsection{Using factory methods}

\subsubsection{Using external meshes}

\subsubsection{Using direct code}

\subsubsection{A simple beam model}

\begin{figure}[ht]
\begin{center}
\iflatexml
 \includegraphics[]{images/FemBeam}
\else
 \includegraphics[width=3.75in]{images/FemBeam}
\fi
\end{center}
\caption{FemBeam model loaded into ArtiSynth.}
\label{FemBeam:fig}
\end{figure}

\subsection{FEM Geometry}

\subsubsection{Surface meshes}

\subsubsection{Embedding geometry within an FEM}

\subsubsection{A beam with an embedded sphere}

% EmbeddedFemSphere

\subsection{Node attachments}
\label{NodeAttachments:sec}

\subsubsection{General principles}

\subsubsection{Connecting a beam to a block}

\begin{figure}[ht]
\begin{center}
\iflatexml
 \includegraphics[]{images/FemBeamWithBlock}
\else
 \includegraphics[width=3.75in]{images/FemBeamWithBlock}
\fi
\end{center}
\caption{FemBeamWithBlock model loaded into ArtiSynth.}
\label{FemBeamWithBlock:fig}
\end{figure}

% FemBeamWithBlock

\subsubsection{Connecting two FEMs together}

% ConnectedFems

\subsection{FEM markers}

\subsubsection{Embedding particles in FEMs}

\subsubsection{Attaching a FEM beam to a muscle}

\begin{figure}[ht]
\begin{center}
\iflatexml
 \includegraphics[]{images/FemBeamWithMuscle}
\else
 \includegraphics[width=3.75in]{images/FemBeamWithMuscle}
\fi
\end{center}
\caption{FemBeamWithMuscle model loaded into ArtiSynth.}
\label{FemBeamWithMuscle:fig}
\end{figure}

% FemBeamWithMuscle

\subsection{Muscle activated FEM models}
\label{sec:fem:muscle}

\subsubsection{FemMuscleModel}

\subsubsection{Activation with fibres}

\subsubsection{Activation with embedded materials}

\subsubsection{Comparision with two beam examples}

\begin{figure}[ht]
\begin{center}
\iflatexml
 \includegraphics[]{images/FemMuscleBeams}
\else
 \includegraphics[width=3.75in]{images/FemMuscleBeams}
\fi
\end{center}
\caption{FemMuscleBeams model loaded into ArtiSynth.}
\label{FemMuscleBeams:fig}
\end{figure}

% FemMuscleBeams

\subsection{Collisions}

\subsubsection{Colliding with FEM geometry}

\subsubsection{Colliding with the surface mesh}

\begin{figure}[ht]
\begin{center}
\iflatexml
 \includegraphics[]{images/FemMuscleHeart}
\else
 \includegraphics[width=3.75in]{images/FemMuscleHeart}
\fi
\end{center}
\caption{FemMuscleHeart model loaded into ArtiSynth.}
\label{FemMuscleHeart:fig}
\end{figure}

% FemMuscleHeart

\subsubsection{Colliding with an embedded sphere}

\begin{figure}[ht]
\begin{center}
\iflatexml
 \includegraphics[]{images/FemCollisions}
\else
 \includegraphics[width=3.75in]{images/FemCollisions}
\fi
\end{center}
\caption{FemCollisions model loaded into ArtiSynth.}
\label{FemCollisions:fig}
\end{figure}

\subsection{Visualization}

stuff stuff stuff

\subsubsection{Rendering settings}

stuff stuff stuff

\subsubsection{Stress and strain plotting}

stuff stuff stuff
